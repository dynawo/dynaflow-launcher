%% Except where otherwise noted, content in this documentation is Copyright (c)
%% 2022, RTE (http://www.rte-france.com) and licensed under a
%% CC-BY-4.0 (https://creativecommons.org/licenses/by/4.0/)
%% license. All rights reserved.

\documentclass[a4paper, 12pt]{report}
% Latex setup
\input{../latex_setup.tex}

\begin{document}

\title{Configure Dynaflow-launcher}
\date\today

\maketitle
\tableofcontents

\chapter[Configure Dynaflow-launcher]{Configure Dynaflow-launcher}

\section{Overview}

The following sections will provide information on:
\begin{itemize}
\item How to create the input files depending on the algorithm of interest;
\item How to configure the input files in order to get the necessary output files to analyse the obtained results;
\end{itemize}


\subsection[Dynaflow-launcher inputs]{Dynaflow-launcher inputs}
\subsubsection{Dynaflow}

To launch Dynaflow, two inputs are required:
\begin{itemize}
  \item a network description file in IIDM format;
  \item a configuration file in json format.
\end{itemize}

The command line used to launch Dynaflow is the following:

\begin{lstlisting}[language=bash, breaklines=true, breakatwhitespace=false, columns=fullflexible]
$> ./myEnvDFL.sh <PATH TO NETWORK FILE> <PATH TO CONFIG FILE>
\end{lstlisting}

for example:

\begin{lstlisting}[language=bash, breaklines=true, breakatwhitespace=false, columns=fullflexible]
$> ./myEnvDFL.sh tests/main/res/TestIIDM_launch.iidm tests/main/res/config_launch.json
\end{lstlisting}

This command will compute the steady-state of the network and provides as
output the constraints and lostEquipments files (see \Dynawo documentation for more details on those files).

\subsubsection{Dynaflow systematic analysis}

The Dynaflow systematic analysis mode can be used to assess the stability of a network test case subject to several contingencies scenarios.

Three inputs are required:
\begin{itemize}
  \item a network description file in IIDM format;
  \item a configuration file in json format;
  \item a contingencies list file in json format.
\end{itemize}

The command line used to launch Dynaflow static analysis is the following:

\begin{lstlisting}[language=bash, breaklines=true, breakatwhitespace=false, columns=fullflexible]
$> ./myEnvDFL.sh <PATH TO NETWORK FILE> <PATH TO CONFIG FILE> <PATH TO CONTINGENCIES FILE>
\end{lstlisting}

for example:

\begin{lstlisting}[language=bash, breaklines=true, breakatwhitespace=false, columns=fullflexible]
$> ./myEnvDFL.sh tests/main_sa/res/TestIIDM_launch.iidm tests/main_sa/res/config_launch.json tests/main_sa/res/contingencies_launch.json
\end{lstlisting}

This command will simulate all the events described in the contingencies file and provide as outputs the constraints and lostEquipments files for each of them.

\textbf{Warning: The systematic analysis assumes that the network used as input is already properly initialized with a steady-state.}

\subsection[Dynaflow-launcher outputs]{Dynaflow-launcher outputs}

Dynaflow-launcher will generate the constraints and lostEquipments files.
When the systematic analysis is launched, one constraints and lostEquipments files will be generated per contingency.

\section[Dynaflow-launcher input files modification]{Dynaflow-launcher input files modification}

This section aims at giving more details on input files in order to help the user parametrize its inputs easily and autonomously.

\subsection{Configuration file}
\label{Dynaflow_Launcher_Configuration_Configuration_File}

The configuration file is a \textbf{.json} file containing the values for the parameters of Dynaflow-launcher. An example is given below:

\lstinputlisting[language=JSON,title=configuration file example]{../resources/syntaxExample/config.json}

The parameters are described below:

\begin{table}[h!]
\center
\begin{tabular}{ l | c | c | c }
\toprule
\textbf{{Name}} & \textbf{{Type}} & \textbf{{Description}} & \textbf{{Default value}}\\
\midrule
\rowcolor{white}
 &  & \small{if true, use infinite reactive limits,} & \\
\rowcolor{white}
\multirow{-2}{*}{\small{InfiniteReactiveLimits}} & \multirow{-2}{*}{\small{boolean}} & \small{if false, use diagrams} & \multirow{-2}{*}{\small{false} } \\
\rowcolor{gray!10}
 &  & \small{Enable static var compensators} &  \\
\rowcolor{gray!10}
\multirow{-2}{*}{\small{SVCRegulationOn}} & \multirow{-2}{*}{\small{boolean}} & \small{regulation} & \multirow{-2}{*}{\small{true}} \\
\rowcolor{white}
\small{ShuntRegulationOn} & \small{boolean} & \small{Enable shunt regulation} & \small{true} \\
\rowcolor{gray!10}
 &  & \small{Enable the automatic research} &  \\
\rowcolor{gray!10}
\multirow{-2}{*}{\small{AutomaticSlackBusOn}} & \multirow{-2}{*}{\small{boolean}} & \small{of the slack node} & \multirow{-2}{*}{\small{true}} \\
\rowcolor{white}
&  & \small{Path to the directory where} &  \\
\rowcolor{white}
\multirow{-2}{*}{\small{OutputDir}}&  \multirow{-2}{*}{\small{string}} & \small{outputs will be written}  &  \multirow{-2}{*}{\small{current path}} \\
\rowcolor{gray!10}
\small{DsoVoltageLevel} & \small{real} & \small{Minimum voltage level of loads} & \small{45} \\
\rowcolor{white}
 & & \small{Active power compensation type:} & \\
\rowcolor{white}
\multirow{-2}{*}{\small{ActivePowerCompensation}} & \multirow{-2}{*}{\small{enum}}& \small{P, TARGET\_P or PMAX} & \multirow{-2}{*}{\small{PMAX}} \\
\rowcolor{gray!10}
\small{StartTime} & \small{integer} & \small{Simulation start time (in s)} & \small{0} \\
\rowcolor{white}
\small{StopTime} & \small{integer} & \small{Simulation stop time (in s)} & \small{100} \\
\rowcolor{gray!10}
\small{TimeStep} & \small{real} & \small{Maximum time solver step value} & \small{10} \\
\bottomrule
\end{tabular}
\caption{Simulation parameters}
\end{table}

The systematic analysis parameters are described below:

\begin{table}[h!]
\center
\begin{tabular}{ l | c | c | c }
\toprule
\textbf{{Name}} & \textbf{{Type}} & \textbf{{Description}} & \textbf{{Default value}}\\
\midrule
\rowcolor{white}
TimeOfEvent & integer & Time when the contingency occurs & 10 \\
\rowcolor{gray!10}
NumberOfThreads & integer & Number of threads to use & 4 \\
\bottomrule
\end{tabular}
\caption{Systematic Analysis parameters}
\end{table}

\subsection{Contingencies file}

The contingency file is a \textbf{.json} file containing a list of elements describing contingencies. An example is given below:
\lstinputlisting[language=JSON,title=configuration file example]{../resources/syntaxExample/contingencies.json}

The id of the element is the id of the node in the IIDM file.
Also, each contingency has an unique id which will be reused in outputs file and logs.
The possible contingency types are the following:
\begin{itemize}
  \item LOAD
  \item GENERATOR
  \item BRANCH
  \item LINE
  \item TWO\_WINDINGS\_TRANSFORMER
  \item THREE\_WINDINGS\_TRANSFORMER
  \item SHUNT\_COMPENSATOR
  \item STATIC\_VAR\_COMPENSATOR
  \item DANGLING\_LINE
  \item HVDC\_LINE
  \item BUSBAR\_SECTION
\end{itemize}

\end{document}

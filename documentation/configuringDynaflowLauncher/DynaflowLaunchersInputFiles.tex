%% Except where otherwise noted, content in this documentation is Copyright (c)
%% 2022, RTE (http://www.rte-france.com) and licensed under a
%% CC-BY-4.0 (https://creativecommons.org/licenses/by/4.0/)
%% license. All rights reserved.

\documentclass[a4paper, 12pt]{report}
% Latex setup
%% Except where otherwise noted, content in this documentation is Copyright (c)
%% 2022, RTE (http://www.rte-france.com) and licensed under a
%% CC-BY-4.0 (https://creativecommons.org/licenses/by/4.0/)
%% license. All rights reserved.

% Latin Modern fam­ily of fonts
\usepackage{lmodern}

\usepackage[english]{babel}

% specify encoding
\usepackage[utf8]{inputenc} % input
\usepackage[T1]{fontenc} % output
\usepackage[table]{xcolor}

% Document structure setup
\usepackage{titlesec} % To change chapter format
\setcounter{tocdepth}{3} % Add subsubsection in Content
\setcounter{secnumdepth}{3} % Add numbering for subsubsection
\setlength{\parindent}{0pt} % No paragraph indentation

% Change title format for chapter
\titleformat{\chapter}{\Huge\bf}{\thechapter}{20pt}{\Huge\bf}

% To add links on page number in Content and hide red rectangle on links
\usepackage[hidelinks, linktoc=all]{hyperref}
\usepackage[nottoc]{tocbibind}  % To add biblio in table of content
\usepackage{textcomp} % For single quote
\usepackage{url} % Allow linebreaks in \url command
\usepackage{listings} % To add code samples

% Default listings parameters
\lstset
{
  aboveskip={1\baselineskip}, % a bit of space above
  backgroundcolor=\color{shadecolor}, % choose the background color
  basicstyle={\ttfamily\footnotesize}, % use font and smaller size \small \footnotesize
  breakatwhitespace=true, % sets if automatic breaks should only happen at whitespace
  breaklines=true, % sets automatic line breaking
  columns=fixed, % nice spacing -> fixed / flexible
  mathescape=false, % escape to latex false
  numbers=left, % where to put the line-numbers
  numberstyle=\tiny\color{gray}, % the style that is used for the line-numbers
  showstringspaces=false, % do not emphasize spaces in strings
  tabsize=4, % number of spaces of a TAB
  texcl=false, % activates or deactivates LaTeX comment lines
  upquote=true % upright quotes
}

% Avoid numbering starting at each chapter for figures
\usepackage{chngcntr}
\counterwithout{figure}{chapter}

\usepackage{tikz} % macro pack­age for cre­at­ing graph­ics
\usepackage{pgfplots} % draws func­tion plots (based on pgf/tikz)
\pgfplotsset{enlarge x limits=false, xlabel={\begin{small}$time$ (s)\end{small}}, height=0.6\textwidth, width=1\textwidth,
yticklabel style={text width={width("$-0.6$")},align=right},
/pgf/number format/precision=4}
\pgfplotstableset{col sep=semicolon}

\usepackage{algorithm} % Add algorithms
\usepackage[noend]{algpseudocode} %  all end ... lines are omitted in algos

\usepackage{amsmath} % Add math­e­mat­i­cal fea­tures
\usepackage{schemabloc} % Add block diagram library (french one)

\usepackage{adjustbox} % Add box for flowchart

\usepackage{booktabs} % for toprule and midrule in tables

\usepackage{tabularx}
\usepackage{multirow}

\usepackage[nolist]{acronym} % don’t write the list of acronyms.
% Acronyms list
\begin{acronym}
\acro{BDF}{Backward Differentiation Formula}
\acro{BE}{Backward Euler}
\acro{DAE}{Differential Algebraic Equations}
\acro{IDA}{Implicit Differential-Algebraic solver}
\acro{LLNL}{Lawrence Livermore National Lab}
\acro{KINSOL}{Krylov Inexact Newton SOLver}
\acro{NR}{Newton-Raphson}
\acro{PLL}{Phase-Locked Loop}
\acro{SVC}{Static Var Compensator}
\acro{SUNDIALS}{SUite of Nonlinear and DIfferential/ALgebraic equation Solvers}
\end{acronym}

% Syntax highlight
%% Except where otherwise noted, content in this documentation is Copyright (c)
%% 2022, RTE (http://www.rte-france.com) and licensed under a
%% CC-BY-4.0 (https://creativecommons.org/licenses/by/4.0/)
%% license. All rights reserved.

\usepackage{color}

\definecolor{blue}{rgb}{0,0,1}
\definecolor{lightblue}{rgb}{.3,.5,1}
\definecolor{darkblue}{rgb}{0,0,.4}
\definecolor{red}{rgb}{1,0,0}
\definecolor{darkred}{rgb}{.56,0,0}
\definecolor{pink}{rgb}{.933,0,.933}
\definecolor{purple}{rgb}{0.58,0,0.82}
\definecolor{green}{rgb}{0.133,0.545,0.133}
\definecolor{darkgreen}{rgb}{0,.4,0}
\definecolor{gray}{rgb}{.3,.3,.3}
\definecolor{darkgray}{rgb}{.2,.2,.2}
\definecolor{shadecolor}{gray}{0.925}

% **********************************************************************************
% Syntax : Bash (bash)
% **********************************************************************************

\lstdefinelanguage{bash}
{
  keywordstyle=\color{blue},
  morekeywords={
    cd,
    export,
    source},
  numbers=none,
  deletekeywords={jobs}
}

% **********************************************************************************
% Syntax : XML
% **********************************************************************************

\lstdefinelanguage{XML}
{
  morestring=[s][\color{purple}]{"}{"},
  morecomment=[s][\color{green}]{<?}{?>},
  morecomment=[s][\color{green}]{<!--}{-->},
  stringstyle=\color{black},
  identifierstyle=\color{blue},
  keywordstyle=\color{red},
  morekeywords={
    xmlns,
    xsi,
    noNamespaceSchemaLocation,
    type,
    source,
    target,
    version,
    tool,
    transRef,
    roleRef,
    objective,
    eventually}
}


% **********************************************************************************
% Syntax : JSON
% **********************************************************************************

\colorlet{punct}{red!60!black}
\definecolor{background}{HTML}{EEEEEE}
\definecolor{delim}{RGB}{20,105,176}
\colorlet{numb}{magenta!60!black}
\lstdefinelanguage{json}{
    basicstyle=\normalfont\ttfamily,
    numbers=left,
    numberstyle=\scriptsize,
    stepnumber=1,
    numbersep=8pt,
    showstringspaces=false,
    breaklines=true,
    frame=lines,
    backgroundcolor=\color{background},
    literate=
     *{0}{{{\color{numb}0}}}{1}
      {1}{{{\color{numb}1}}}{1}
      {2}{{{\color{numb}2}}}{1}
      {3}{{{\color{numb}3}}}{1}
      {4}{{{\color{numb}4}}}{1}
      {5}{{{\color{numb}5}}}{1}
      {6}{{{\color{numb}6}}}{1}
      {7}{{{\color{numb}7}}}{1}
      {8}{{{\color{numb}8}}}{1}
      {9}{{{\color{numb}9}}}{1}
      {:}{{{\color{punct}{:}}}}{1}
      {,}{{{\color{punct}{,}}}}{1}
      {\{}{{{\color{delim}{\{}}}}{1}
      {\}}{{{\color{delim}{\}}}}}{1}
      {[}{{{\color{delim}{[}}}}{1}
      {]}{{{\color{delim}{]}}}}{1},
}

% **********************************************************************************
% Syntax : Modelica (modelica)
% **********************************************************************************
\lstdefinelanguage{Modelica}{
  alsoletter={...},
  morekeywords=[1]{ % types
      Boolean,
      Integer,
      Real},
  keywordstyle=[1]\color{red},
  morekeywords=[2]{ % keywords
    algorithm,
    and,
    annotation,
    assert,
    block,
    class,
    connector,
    constant,
    discrete,
    else,
    elseif,
    elsewhen,
    end,
    equation,
    exit,
    extends,
    external,
    false,
    final,
    flow,
    for,
    function,
    if,
    in,
    inner,
    input,
    import,
    loop,
    model,
    nondiscrete,
    not,
    or,
    outer,
    output,
    package,
    parameter,
    public,
    protected,
    record,
    redeclare,
    replaceable,
    return,
    size,
    terminate,
    then,
    true,
    type,
    when,
    while},
  keywordstyle=[2]\color{darkred},
  morekeywords=[3]{ % functions
    abs,
    acos,
    asin,
    atan,
    atan2,
    Complex,
    connect,
    conj,
    cos,
    cosh,
    cross,
    der,
    edge,
    exp,
    fromPolar,
    imag,
    noEvent,
    pre,
    sign,
    sin,
    sinh,
    sqrt,
    tan,
    tanh},
  keywordstyle=[3]\color{blue},
  morecomment=[l][\color{green}]{//}, % comments
  morecomment=[s][\color{green}]{/*}{*/}, % comments
  morestring=[b][\color{pink}]{'}, % strings
  morestring=[b][\color{pink}]{"}, % strings
}

\usepackage{tikz}
\definecolor{blue}{rgb}{.3,.5,1}
\definecolor{red}{rgb}{1,0,0}
\usetikzlibrary{shapes,arrows}
% Define block styles
\tikzstyle{decision} = [diamond, draw, fill=blue!20,
    text width=4.5em, text badly centered, node distance=3cm, inner sep=0pt]
\tikzstyle{block} = [rectangle, draw, fill=blue!20,
    text width=5em, text centered, rounded corners, minimum height=4em]
\tikzstyle{line} = [draw, -latex']
\tikzstyle{cloud} = [draw, ellipse,fill=red!20, node distance=3cm,
    minimum height=2em]
    \usetikzlibrary{calc}


\usepackage{xspace} % Define typography
\usepackage{dirtree}
\newcommand{\Dynawo}[0]{Dyna$\omega$o\xspace}


\usepackage{float}

\begin{document}

\title{Configure Dynaflow-launcher}
\date\today

\maketitle
\tableofcontents

\chapter[Configure Dynaflow-launcher]{Configure Dynaflow-launcher}

\section{Overview}

The following sections will provide information on:
\begin{itemize}
\item How to create the input files depending on the algorithm of interest;
\item How to configure the input files in order to get the necessary output files to analyse the obtained results;
\end{itemize}


\subsection[Dynaflow-launcher inputs]{Dynaflow-launcher inputs}
\subsubsection{Dynaflow}

To launch Dynaflow, two inputs are required:
\begin{itemize}
  \item a network description file in IIDM format;
  \item a configuration file in json format.
\end{itemize}

The command line used to launch Dynaflow is the following:

\begin{lstlisting}[language=bash, breaklines=true, breakatwhitespace=false, columns=fullflexible]
$> ./myEnvDFL.sh launch <PATH TO NETWORK FILE> <PATH TO CONFIG FILE>
\end{lstlisting}

for example:

\begin{lstlisting}[language=bash, breaklines=true, breakatwhitespace=false, columns=fullflexible]
$> ./myEnvDFL.sh launch tests/main/res/TestIIDM_launch.iidm tests/main/res/config_launch.json
\end{lstlisting}

This command will compute the steady-state of the network and provides as
output the constraints and lostEquipments files (see \Dynawo documentation for more details on those files).\newline

You can also provide a zip archive as input. In this case, you need to specify the path of the zip archive along with the paths of the unzipped files :

\begin{lstlisting}[language=bash, breaklines=true, breakatwhitespace=false, columns=fullflexible]
$> ./myEnvDFL.sh launch tests/main/res/TestIIDM_launch.iidm tests/main/res/config_launch.json --input-archive tests/main/archive.zip
\end{lstlisting}

\subsubsection{Dynaflow systematic analysis}

The Dynaflow systematic analysis mode can be used to assess the stability of a network test case subject to several contingencies scenarios.

Three inputs are required:
\begin{itemize}
  \item a network description file in IIDM format;
  \item a configuration file in json format;
  \item a contingencies list file in json format.
\end{itemize}

The command line used to launch Dynaflow static analysis is the following:

\begin{lstlisting}[language=bash, breaklines=true, breakatwhitespace=false, columns=fullflexible]
$> ./myEnvDFL.sh launch-sa <PATH TO NETWORK FILE> <PATH TO CONFIG FILE> <PATH TO CONTINGENCIES FILE>
\end{lstlisting}

for example:

\begin{lstlisting}[language=bash, breaklines=true, breakatwhitespace=false, columns=fullflexible]
$> ./myEnvDFL.sh launch-sa tests/main_sa/res/TestIIDM_launch.iidm tests/main_sa/res/config_launch.json tests/main_sa/res/contingencies_launch.json
\end{lstlisting}

This command will simulate all the events described in the contingencies file and provide as outputs the constraints and lostEquipments files for each of them.

\textbf{Warning: The systematic analysis assumes that the network used as input is already properly initialized with a steady-state. To skip this initialization process, refer to the next section.}

\subsubsection{Dynaflow steady-state and security analysis workflow}

The systematic analysis assumes that the network used as input is already initialized with a steady-state.
To automatically perform the steady-state initialization followed by a systematic analysis, provide the same inputs as for a systematic analysis and specify the \texttt{-{}-nsa} option.

For example:

\begin{lstlisting}[language=bash, breaklines=true, breakatwhitespace=false, columns=fullflexible]
  $> ./myEnvDFL.sh launch-nsa tests/main_sa/res/TestIIDM_launch.iidm tests/main_sa/res/config_launch.json tests/main_sa/res/contingencies_launch.json
\end{lstlisting}

\subsection[Dynaflow-launcher outputs]{Dynaflow-launcher outputs}

Dynaflow-launcher will generate the constraints and lostEquipments files.
When the systematic analysis is launched, one constraints and lostEquipments files will be generated per contingency.

\section[Dynaflow-launcher input files modification]{Dynaflow-launcher input files modification}

This section aims at giving more details on input files in order to help the user parametrize its inputs easily and autonomously.

\subsection{Configuration file}
\label{Dynaflow_Launcher_Configuration_Configuration_File}

The configuration file is a \textbf{.json} file containing the values for the parameters of Dynaflow-launcher. An example is given below:

\subsubsection{Simple configuration}

\lstinputlisting[language=JSON,title=configuration file example]{../resources/syntaxExample/config.json}

The parameters are described below:

\begin{table}[H]
\center
\begin{tabular}{ l | c | c | c }
\toprule
\textbf{{Name}} & \textbf{{Type}} & \textbf{{Description}} & \textbf{{Default value}}\\
\midrule
\rowcolor{white}
 &  & \small{if true, use infinite reactive limits,} & \\
\rowcolor{white}
\multirow{-2}{*}{\small{InfiniteReactiveLimits}} & \multirow{-2}{*}{\small{boolean}} & \small{if false, use diagrams} & \multirow{-2}{*}{\small{false} } \\
\rowcolor{gray!10}
 &  & \small{Enable static var compensators} &  \\
\rowcolor{gray!10}
\multirow{-2}{*}{\small{SVCRegulationOn}} & \multirow{-2}{*}{\small{boolean}} & \small{regulation} & \multirow{-2}{*}{\small{true}} \\
\rowcolor{white}
\small{ShuntRegulationOn} & \small{boolean} & \small{Enable shunt regulation} & \small{true} \\
\rowcolor{gray!10}
 &  & \small{Enable the automatic research} &  \\
\rowcolor{gray!10}
\multirow{-2}{*}{\small{AutomaticSlackBusOn}} & \multirow{-2}{*}{\small{boolean}} & \small{of the slack node} & \multirow{-2}{*}{\small{true}} \\
\rowcolor{white}
&  & \small{Path to the directory where} &  \\
\rowcolor{white}
\multirow{-2}{*}{\small{OutputDir}}&  \multirow{-2}{*}{\small{string}} & \small{outputs will be written}  &  \multirow{-2}{*}{\small{current path}} \\
\rowcolor{gray!10}
\small{DsoVoltageLevel} & \small{real} & \small{Minimum voltage level of loads} & \small{45}
\\\rowcolor{white}
&  & \small{Maximum voltage level for which} &  \\
\rowcolor{white}
& & \small{generator transformers are}  &   \\
\rowcolor{white}
\multirow{-3}{*}{\small{TfoVoltageLevel}}&  \multirow{-3}{*}{\small{real}} & \small{considered to be in the iidm file}  &  \multirow{-3}{*}{\small{100}} \\
\rowcolor{gray!10}
 & & \small{Active power compensation type:} & \\
\rowcolor{gray!10}
\multirow{-2}{*}{\small{ActivePowerCompensation}} & \multirow{-2}{*}{\small{enum}}& \small{P, targetP or PMax} & \multirow{-2}{*}{\small{PMax}} \\
\rowcolor{white}
\small{StartTime} & \small{integer} & \small{Simulation start time (in s)} & \small{0} \\
\rowcolor{gray!10}
\small{StopTime} & \small{integer} & \small{Simulation stop time (in s)} & \small{100} \\
\rowcolor{white}
\small{TimeStep} & \small{real} & \small{Maximum time solver step value} & \small{10} \\
\rowcolor{gray!10}
\small{MinTimeStep} & \small{real} & \small{Minimum time solver step value} & \small{1} \\
\rowcolor{white}
& & & \small{Depends on the} \\
\rowcolor{white}
& & & \small{buildtype and the} \\
\rowcolor{white}
\multirow{-3}{*}{\small{ChosenOutputs}} & \multirow{-3}{*}{\small{struct}}  & \multirow{-3}{*}{\small{Chosen outputs}} & \small{simulation kind} \\
\rowcolor{gray!10}
\small{StartingPointMode} & \small{string} & \small{Starting point mode : warm / flat} & \small{warm} \\
\rowcolor{white}
\small{Precision} & \small{real} & \small{real number precision} & \small{\Dynawo default} \\
\rowcolor{gray!10}
\small{AssemblingPath} & \small{string} & \small{Path to assembling file (see \ref{DFL_Dyn_Models})} & \small{None} \\
\rowcolor{white}
\small{SettingPath} & \small{string} & \small{Path to setting file (see \ref{DFL_Dyn_Models})} & \small{None} \\
\bottomrule
\end{tabular}
\caption{Simulation parameters}
\label{DFL_Simu_param}
\end{table}

\begin{table}[H]
The default chosen outputs depend on the build type (Release or Debug) and the simulation kind (Steady State or Security Analysis). The \textit{ChosenOutput} attribute array in the configuration file can add optional chosen outputs.
\center
\begin{tabular}{| c | c | c |}
\cline{2-3}
\multicolumn{1}{c|}{} & Steady State & Security Analysis \\
\hline
\multirow{2}{*}{Release} & \textcolor{green}{\small{IIDM} \textcolor{gray}{CONSTRAINTS}} & \small{\textcolor{gray}{IIDM} \textcolor{green}{CONSTRAINTS}} \\
& \small{\textcolor{gray}{LOSTEQ}} \textcolor{gray}{TIMELINE} & \small{\textcolor{green}{LOSTEQ} \textcolor{gray}{TIMELINE}} \\
\hline
\multirow{2}{*}{Debug} & \textcolor{green}{\small{IIDM CONSTRAINTS}} & \textcolor{green}{\small{IIDM CONSTRAINTS}} \\
& \textcolor{green}{\small{LOSTEQ TIMELINE}} & \textcolor{green}{\small{LOSTEQ TIMELINE}} \\
\hline
\end{tabular}
\caption{Chosen Outputs depending on the buildtype and the simulation kind : outputs chosen by default are \textcolor{green}{green} and optional inputs are \textcolor{gray}{gray}}
\end{table}

The \textit{StartingPointMode} indicates the starting point values considered in the simulation. If it's \textit{warm}, the starting values for voltage, phase and injections are considered previously calculated.
If, instead, \textit{StartingPointMode} is set to \textit{flat}, starting point values considered are nominal value for bus voltages and set points values for injections.

\begin{table}[H]
The systematic analysis parameters are described below:
\center
\begin{tabular}{ l | c | c | c }
\toprule
\textbf{{Name}} & \textbf{{Type}} & \textbf{{Description}} & \textbf{{Default value}}\\
\midrule
\rowcolor{white}
TimeOfEvent & integer & Time when the contingency occurs & 10 \\
\rowcolor{gray!10}
StartingDumpFile & string & Path to a \Dynawo dump file &  \\
\bottomrule
\end{tabular}
\caption{Systematic Analysis parameters}
\end{table}


\subsubsection{Apply a different configuration in security analysis}

It is possible to have a different configuration for security analysis by adding any parameter from table \ref{DFL_Simu_param} in the \textit{sa} array.
The parameters contained in the \textit{sa} array will then override the parameters for N in the configuration file.
If no parameters were previously specified for N, it will override the default parameters of the simulation.
\textbf{The only parameter not eligible for this override is 'outputDir'.}

\lstinputlisting[language=JSON,title=configuration file example for N and SA simulation]{../resources/syntaxExample/config_SA.json}


\subsection{Contingencies file}

The contingency file is a \textbf{.json} file containing a list of elements describing contingencies. An example is given below:
\lstinputlisting[language=JSON,title=configuration file example]{../resources/syntaxExample/contingencies.json}

The id of the element is the id of the node in the IIDM file.
Also, each contingency has an unique id which will be reused in outputs file and logs.
The possible contingency types are the following:
\begin{itemize}
  \item LOAD
  \item GENERATOR
  \item BRANCH
  \item LINE
  \item TWO\_WINDINGS\_TRANSFORMER
  \item THREE\_WINDINGS\_TRANSFORMER
  \item SHUNT\_COMPENSATOR
  \item STATIC\_VAR\_COMPENSATOR
  \item DANGLING\_LINE
  \item HVDC\_LINE
  \item BUSBAR\_SECTION
\end{itemize}

\end{document}

%% Except where otherwise noted, content in this documentation is Copyright (c)
%% 2022, RTE (http://www.rte-france.com) and licensed under a
%% CC-BY-4.0 (https://creativecommons.org/licenses/by/4.0/)
%% license. All rights reserved.

\documentclass[a4paper, 12pt]{report}

% Latex setup
%% Except where otherwise noted, content in this documentation is Copyright (c)
%% 2022, RTE (http://www.rte-france.com) and licensed under a
%% CC-BY-4.0 (https://creativecommons.org/licenses/by/4.0/)
%% license. All rights reserved.

% Latin Modern fam­ily of fonts
\usepackage{lmodern}

\usepackage[english]{babel}

% specify encoding
\usepackage[utf8]{inputenc} % input
\usepackage[T1]{fontenc} % output
\usepackage[table]{xcolor}

% Document structure setup
\usepackage{titlesec} % To change chapter format
\setcounter{tocdepth}{3} % Add subsubsection in Content
\setcounter{secnumdepth}{3} % Add numbering for subsubsection
\setlength{\parindent}{0pt} % No paragraph indentation

% Change title format for chapter
\titleformat{\chapter}{\Huge\bf}{\thechapter}{20pt}{\Huge\bf}

% To add links on page number in Content and hide red rectangle on links
\usepackage[hidelinks, linktoc=all]{hyperref}
\usepackage[nottoc]{tocbibind}  % To add biblio in table of content
\usepackage{textcomp} % For single quote
\usepackage{url} % Allow linebreaks in \url command
\usepackage{listings} % To add code samples

% Default listings parameters
\lstset
{
  aboveskip={1\baselineskip}, % a bit of space above
  backgroundcolor=\color{shadecolor}, % choose the background color
  basicstyle={\ttfamily\footnotesize}, % use font and smaller size \small \footnotesize
  breakatwhitespace=true, % sets if automatic breaks should only happen at whitespace
  breaklines=true, % sets automatic line breaking
  columns=fixed, % nice spacing -> fixed / flexible
  mathescape=false, % escape to latex false
  numbers=left, % where to put the line-numbers
  numberstyle=\tiny\color{gray}, % the style that is used for the line-numbers
  showstringspaces=false, % do not emphasize spaces in strings
  tabsize=4, % number of spaces of a TAB
  texcl=false, % activates or deactivates LaTeX comment lines
  upquote=true % upright quotes
}

% Avoid numbering starting at each chapter for figures
\usepackage{chngcntr}
\counterwithout{figure}{chapter}

\usepackage{tikz} % macro pack­age for cre­at­ing graph­ics
\usepackage{pgfplots} % draws func­tion plots (based on pgf/tikz)
\pgfplotsset{enlarge x limits=false, xlabel={\begin{small}$time$ (s)\end{small}}, height=0.6\textwidth, width=1\textwidth,
yticklabel style={text width={width("$-0.6$")},align=right},
/pgf/number format/precision=4}
\pgfplotstableset{col sep=semicolon}

\usepackage{algorithm} % Add algorithms
\usepackage[noend]{algpseudocode} %  all end ... lines are omitted in algos

\usepackage{amsmath} % Add math­e­mat­i­cal fea­tures
\usepackage{schemabloc} % Add block diagram library (french one)

\usepackage{adjustbox} % Add box for flowchart

\usepackage{booktabs} % for toprule and midrule in tables

\usepackage{tabularx}
\usepackage{multirow}

\usepackage[nolist]{acronym} % don’t write the list of acronyms.
% Acronyms list
\begin{acronym}
\acro{BDF}{Backward Differentiation Formula}
\acro{BE}{Backward Euler}
\acro{DAE}{Differential Algebraic Equations}
\acro{IDA}{Implicit Differential-Algebraic solver}
\acro{LLNL}{Lawrence Livermore National Lab}
\acro{KINSOL}{Krylov Inexact Newton SOLver}
\acro{NR}{Newton-Raphson}
\acro{PLL}{Phase-Locked Loop}
\acro{SVC}{Static Var Compensator}
\acro{SUNDIALS}{SUite of Nonlinear and DIfferential/ALgebraic equation Solvers}
\end{acronym}

% Syntax highlight
%% Except where otherwise noted, content in this documentation is Copyright (c)
%% 2022, RTE (http://www.rte-france.com) and licensed under a
%% CC-BY-4.0 (https://creativecommons.org/licenses/by/4.0/)
%% license. All rights reserved.

\usepackage{color}

\definecolor{blue}{rgb}{0,0,1}
\definecolor{lightblue}{rgb}{.3,.5,1}
\definecolor{darkblue}{rgb}{0,0,.4}
\definecolor{red}{rgb}{1,0,0}
\definecolor{darkred}{rgb}{.56,0,0}
\definecolor{pink}{rgb}{.933,0,.933}
\definecolor{purple}{rgb}{0.58,0,0.82}
\definecolor{green}{rgb}{0.133,0.545,0.133}
\definecolor{darkgreen}{rgb}{0,.4,0}
\definecolor{gray}{rgb}{.3,.3,.3}
\definecolor{darkgray}{rgb}{.2,.2,.2}
\definecolor{shadecolor}{gray}{0.925}

% **********************************************************************************
% Syntax : Bash (bash)
% **********************************************************************************

\lstdefinelanguage{bash}
{
  keywordstyle=\color{blue},
  morekeywords={
    cd,
    export,
    source},
  numbers=none,
  deletekeywords={jobs}
}

% **********************************************************************************
% Syntax : XML
% **********************************************************************************

\lstdefinelanguage{XML}
{
  morestring=[s][\color{purple}]{"}{"},
  morecomment=[s][\color{green}]{<?}{?>},
  morecomment=[s][\color{green}]{<!--}{-->},
  stringstyle=\color{black},
  identifierstyle=\color{blue},
  keywordstyle=\color{red},
  morekeywords={
    xmlns,
    xsi,
    noNamespaceSchemaLocation,
    type,
    source,
    target,
    version,
    tool,
    transRef,
    roleRef,
    objective,
    eventually}
}


% **********************************************************************************
% Syntax : JSON
% **********************************************************************************

\colorlet{punct}{red!60!black}
\definecolor{background}{HTML}{EEEEEE}
\definecolor{delim}{RGB}{20,105,176}
\colorlet{numb}{magenta!60!black}
\lstdefinelanguage{json}{
    basicstyle=\normalfont\ttfamily,
    numbers=left,
    numberstyle=\scriptsize,
    stepnumber=1,
    numbersep=8pt,
    showstringspaces=false,
    breaklines=true,
    frame=lines,
    backgroundcolor=\color{background},
    literate=
     *{0}{{{\color{numb}0}}}{1}
      {1}{{{\color{numb}1}}}{1}
      {2}{{{\color{numb}2}}}{1}
      {3}{{{\color{numb}3}}}{1}
      {4}{{{\color{numb}4}}}{1}
      {5}{{{\color{numb}5}}}{1}
      {6}{{{\color{numb}6}}}{1}
      {7}{{{\color{numb}7}}}{1}
      {8}{{{\color{numb}8}}}{1}
      {9}{{{\color{numb}9}}}{1}
      {:}{{{\color{punct}{:}}}}{1}
      {,}{{{\color{punct}{,}}}}{1}
      {\{}{{{\color{delim}{\{}}}}{1}
      {\}}{{{\color{delim}{\}}}}}{1}
      {[}{{{\color{delim}{[}}}}{1}
      {]}{{{\color{delim}{]}}}}{1},
}

% **********************************************************************************
% Syntax : Modelica (modelica)
% **********************************************************************************
\lstdefinelanguage{Modelica}{
  alsoletter={...},
  morekeywords=[1]{ % types
      Boolean,
      Integer,
      Real},
  keywordstyle=[1]\color{red},
  morekeywords=[2]{ % keywords
    algorithm,
    and,
    annotation,
    assert,
    block,
    class,
    connector,
    constant,
    discrete,
    else,
    elseif,
    elsewhen,
    end,
    equation,
    exit,
    extends,
    external,
    false,
    final,
    flow,
    for,
    function,
    if,
    in,
    inner,
    input,
    import,
    loop,
    model,
    nondiscrete,
    not,
    or,
    outer,
    output,
    package,
    parameter,
    public,
    protected,
    record,
    redeclare,
    replaceable,
    return,
    size,
    terminate,
    then,
    true,
    type,
    when,
    while},
  keywordstyle=[2]\color{darkred},
  morekeywords=[3]{ % functions
    abs,
    acos,
    asin,
    atan,
    atan2,
    Complex,
    connect,
    conj,
    cos,
    cosh,
    cross,
    der,
    edge,
    exp,
    fromPolar,
    imag,
    noEvent,
    pre,
    sign,
    sin,
    sinh,
    sqrt,
    tan,
    tanh},
  keywordstyle=[3]\color{blue},
  morecomment=[l][\color{green}]{//}, % comments
  morecomment=[s][\color{green}]{/*}{*/}, % comments
  morestring=[b][\color{pink}]{'}, % strings
  morestring=[b][\color{pink}]{"}, % strings
}

\usepackage{tikz}
\definecolor{blue}{rgb}{.3,.5,1}
\definecolor{red}{rgb}{1,0,0}
\usetikzlibrary{shapes,arrows}
% Define block styles
\tikzstyle{decision} = [diamond, draw, fill=blue!20,
    text width=4.5em, text badly centered, node distance=3cm, inner sep=0pt]
\tikzstyle{block} = [rectangle, draw, fill=blue!20,
    text width=5em, text centered, rounded corners, minimum height=4em]
\tikzstyle{line} = [draw, -latex']
\tikzstyle{cloud} = [draw, ellipse,fill=red!20, node distance=3cm,
    minimum height=2em]
    \usetikzlibrary{calc}


\usepackage{xspace} % Define typography
\usepackage{dirtree}
\newcommand{\Dynawo}[0]{Dyna$\omega$o\xspace}


\begin{document}

\title{Dynaflow-launcher Functional Documentation}
\date\today

\maketitle
\tableofcontents

\chapter{Functional documentation}

\section[Dynaflow-launcher Overview]{Dynaflow-launcher Overview}
Dynaflow-launcher creates, for a given .iidm network file and config.json, all the input files needed for a DynaFlow simulation, runs a simulation and collects the results.
\par In particular, it generates the .dyd file containing all the modelling choices for all the system's elements and the .par file that contains all the parameters for these models.
\par The next section presents the modelling choices that are made.

\section{Modelling}

\subsection{Loads}
The models that are used for the loads depend on the parameter DsoVoltageLevel in the config.json file.
\par Loads that are connected to buses with a voltage level below DsoVoltageLevel are modelled as fixed PQ loads. Nothing is put in the .dyd file as it is the default c++ model in \Dynawo.
\par Loads that are connected to buses with a voltage level above DsoVoltageLevel are modelled as restorative loads with voltage limits (DYNModelLoadRestorativeWithLimits). After a disturbance, these loads go back to their initial P and Q after a transient, unless the voltage at their bus is outside the voltage limits, in which case they behave as alpha-beta loads (with a voltage-dependent behaviour).

\subsection{Generators}
First, the parameter InfiniteReactiveLimits in the config.json determines whether the generator models have infinite reactive power limits or PQ diagrams.

\par The second parameter that has an influence is ActivePowerCompensation. It determines whether the generators participate in the active power balancing proportionally to PMax, P or PTarget. These two parameters are chosen by the user.

\par The other choices are made depending on the grid situation and are automatically made by the tool.

\par The generator can regulate the voltage or not. This information is given by the .iidm file. If a generator does not participate in the voltage regulation, it is also considered that it doesn't participate in the active power balancing. Therefore, no model is put in the .dyd and the default PQ c++ model is used.

\par It can participate in the active power balancing or not, depending on its active power reference. Generators with an active power target equal to 0 do not participate in the active power balancing. They can however regulate the voltage.

\par Concerning the voltage regulation:

\begin{itemize}
\item If a generator regulates the voltage at its connection bus, GeneratorPVDiagramPQSignalN (or GeneratorPVSignalN if infinite reactive limits are used) is used. This model strictly regulates the bus voltage, unless the reactive limits are hit, in which case Q = QMin or Q = QMax. The signal N is used for the active power balancing (it is not considered if the generator does not participate in the active power balancing).
\item If it regulates the voltage at a distant bus, GeneratorPVRemoteDiagramPQSignalN (or GeneratorPVRemoteSignalN) is used.
\item If another generator regulates the voltage at the same node (connection node or distant), then GeneratorPQPropDiagramPQSignalN (or GeneratorPQPropSignalN) is used for both generators, along with a VRRemote model associated to this bus to handle the coordinated voltage regulation.
\end{itemize}

\subsection{HVDC}
\par Two types of HVDC links are modelled in DynaFlow : LCC with HvdcPTanPhi models and VSC with HvdcPV and HvdcPQProp models. Models with PQ diagrams or infinite reactive power limits are used depending on the choice made in the config.json.

\par For the VSCs, the same logic is applied regarding the voltage regulation to choose between HvdcPV or HvdcPQProp models.

\par For all HVDC links a specific "Dangling" model is used if the HVDC connects two different connected components.

\par For example, for two HVDC VSC links in parallel regulating the same buses with PQ diagram taken into account, the model HvdcPQPropDiagramPQ is used. For an LCC connecting two different connected components, the model HvdcPTanPhiDangling is used.

\par Moreover, for HVDC VSC links, if an AC emulation law is described in the .iidm file with the proper extension, it is taken into account in the model, with the "Emulation" suffix.

\subsection{Static Var Compensators}
\par For Static Var Compensators, StaticVarCompensatorPV are used if the SVarC strictly regulates the voltage and StaticVarCompensatorPVProp if an U + Lambda*Q control law is used, with a "Remote" suffix if the regulated bus is distant. If a special extension is present in the .iidm describing the control law of the mode of the SVarC, then other models are used, with the "ModeHandling" suffix.

\par For example, an SVarC regulating the voltage at a distant bus with an U + Lambda*Q control law and with the proper extension describing the mode handling in the .iidm, is modelled by StaticVarCompensatorPVPropRemoteModeHandling.

\section{Solver}

The solver used is the Simplified Solver (SolverSIM) from \Dynawo.
The Maximum time step value is 10s.

The full configuration used is the following:

\lstinputlisting[language=XML,title=\Dynawo Simplified Solver configuration]{../../etc/solver.par}

\end{document}

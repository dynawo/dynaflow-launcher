%% Except where otherwise noted, content in this documentation is Copyright (c)
%% 2022, RTE (http://www.rte-france.com) and licensed under a
%% CC-BY-4.0 (https://creativecommons.org/licenses/by/4.0/)
%% license. All rights reserved.

\documentclass[a4paper, 12pt]{report}

% Latex setup
%% Except where otherwise noted, content in this documentation is Copyright (c)
%% 2022, RTE (http://www.rte-france.com) and licensed under a
%% CC-BY-4.0 (https://creativecommons.org/licenses/by/4.0/)
%% license. All rights reserved.

% Latin Modern fam­ily of fonts
\usepackage{lmodern}

\usepackage[english]{babel}

% specify encoding
\usepackage[utf8]{inputenc} % input
\usepackage[T1]{fontenc} % output
\usepackage[table]{xcolor}

% Document structure setup
\usepackage{titlesec} % To change chapter format
\setcounter{tocdepth}{3} % Add subsubsection in Content
\setcounter{secnumdepth}{3} % Add numbering for subsubsection
\setlength{\parindent}{0pt} % No paragraph indentation

% Change title format for chapter
\titleformat{\chapter}{\Huge\bf}{\thechapter}{20pt}{\Huge\bf}

% To add links on page number in Content and hide red rectangle on links
\usepackage[hidelinks, linktoc=all]{hyperref}
\usepackage[nottoc]{tocbibind}  % To add biblio in table of content
\usepackage{textcomp} % For single quote
\usepackage{url} % Allow linebreaks in \url command
\usepackage{listings} % To add code samples

% Default listings parameters
\lstset
{
  aboveskip={1\baselineskip}, % a bit of space above
  backgroundcolor=\color{shadecolor}, % choose the background color
  basicstyle={\ttfamily\footnotesize}, % use font and smaller size \small \footnotesize
  breakatwhitespace=true, % sets if automatic breaks should only happen at whitespace
  breaklines=true, % sets automatic line breaking
  columns=fixed, % nice spacing -> fixed / flexible
  mathescape=false, % escape to latex false
  numbers=left, % where to put the line-numbers
  numberstyle=\tiny\color{gray}, % the style that is used for the line-numbers
  showstringspaces=false, % do not emphasize spaces in strings
  tabsize=4, % number of spaces of a TAB
  texcl=false, % activates or deactivates LaTeX comment lines
  upquote=true % upright quotes
}

% Avoid numbering starting at each chapter for figures
\usepackage{chngcntr}
\counterwithout{figure}{chapter}

\usepackage{tikz} % macro pack­age for cre­at­ing graph­ics
\usepackage{pgfplots} % draws func­tion plots (based on pgf/tikz)
\pgfplotsset{enlarge x limits=false, xlabel={\begin{small}$time$ (s)\end{small}}, height=0.6\textwidth, width=1\textwidth,
yticklabel style={text width={width("$-0.6$")},align=right},
/pgf/number format/precision=4}
\pgfplotstableset{col sep=semicolon}

\usepackage{algorithm} % Add algorithms
\usepackage[noend]{algpseudocode} %  all end ... lines are omitted in algos

\usepackage{amsmath} % Add math­e­mat­i­cal fea­tures
\usepackage{schemabloc} % Add block diagram library (french one)

\usepackage{adjustbox} % Add box for flowchart

\usepackage{booktabs} % for toprule and midrule in tables

\usepackage{tabularx}
\usepackage{multirow}

\usepackage[nolist]{acronym} % don’t write the list of acronyms.
% Acronyms list
\begin{acronym}
\acro{BDF}{Backward Differentiation Formula}
\acro{BE}{Backward Euler}
\acro{DAE}{Differential Algebraic Equations}
\acro{IDA}{Implicit Differential-Algebraic solver}
\acro{LLNL}{Lawrence Livermore National Lab}
\acro{KINSOL}{Krylov Inexact Newton SOLver}
\acro{NR}{Newton-Raphson}
\acro{PLL}{Phase-Locked Loop}
\acro{SVC}{Static Var Compensator}
\acro{SUNDIALS}{SUite of Nonlinear and DIfferential/ALgebraic equation Solvers}
\end{acronym}

% Syntax highlight
%% Except where otherwise noted, content in this documentation is Copyright (c)
%% 2022, RTE (http://www.rte-france.com) and licensed under a
%% CC-BY-4.0 (https://creativecommons.org/licenses/by/4.0/)
%% license. All rights reserved.

\usepackage{color}

\definecolor{blue}{rgb}{0,0,1}
\definecolor{lightblue}{rgb}{.3,.5,1}
\definecolor{darkblue}{rgb}{0,0,.4}
\definecolor{red}{rgb}{1,0,0}
\definecolor{darkred}{rgb}{.56,0,0}
\definecolor{pink}{rgb}{.933,0,.933}
\definecolor{purple}{rgb}{0.58,0,0.82}
\definecolor{green}{rgb}{0.133,0.545,0.133}
\definecolor{darkgreen}{rgb}{0,.4,0}
\definecolor{gray}{rgb}{.3,.3,.3}
\definecolor{darkgray}{rgb}{.2,.2,.2}
\definecolor{shadecolor}{gray}{0.925}

% **********************************************************************************
% Syntax : Bash (bash)
% **********************************************************************************

\lstdefinelanguage{bash}
{
  keywordstyle=\color{blue},
  morekeywords={
    cd,
    export,
    source},
  numbers=none,
  deletekeywords={jobs}
}

% **********************************************************************************
% Syntax : XML
% **********************************************************************************

\lstdefinelanguage{XML}
{
  morestring=[s][\color{purple}]{"}{"},
  morecomment=[s][\color{green}]{<?}{?>},
  morecomment=[s][\color{green}]{<!--}{-->},
  stringstyle=\color{black},
  identifierstyle=\color{blue},
  keywordstyle=\color{red},
  morekeywords={
    xmlns,
    xsi,
    noNamespaceSchemaLocation,
    type,
    source,
    target,
    version,
    tool,
    transRef,
    roleRef,
    objective,
    eventually}
}


% **********************************************************************************
% Syntax : JSON
% **********************************************************************************

\colorlet{punct}{red!60!black}
\definecolor{background}{HTML}{EEEEEE}
\definecolor{delim}{RGB}{20,105,176}
\colorlet{numb}{magenta!60!black}
\lstdefinelanguage{json}{
    basicstyle=\normalfont\ttfamily,
    numbers=left,
    numberstyle=\scriptsize,
    stepnumber=1,
    numbersep=8pt,
    showstringspaces=false,
    breaklines=true,
    frame=lines,
    backgroundcolor=\color{background},
    literate=
     *{0}{{{\color{numb}0}}}{1}
      {1}{{{\color{numb}1}}}{1}
      {2}{{{\color{numb}2}}}{1}
      {3}{{{\color{numb}3}}}{1}
      {4}{{{\color{numb}4}}}{1}
      {5}{{{\color{numb}5}}}{1}
      {6}{{{\color{numb}6}}}{1}
      {7}{{{\color{numb}7}}}{1}
      {8}{{{\color{numb}8}}}{1}
      {9}{{{\color{numb}9}}}{1}
      {:}{{{\color{punct}{:}}}}{1}
      {,}{{{\color{punct}{,}}}}{1}
      {\{}{{{\color{delim}{\{}}}}{1}
      {\}}{{{\color{delim}{\}}}}}{1}
      {[}{{{\color{delim}{[}}}}{1}
      {]}{{{\color{delim}{]}}}}{1},
}

% **********************************************************************************
% Syntax : Modelica (modelica)
% **********************************************************************************
\lstdefinelanguage{Modelica}{
  alsoletter={...},
  morekeywords=[1]{ % types
      Boolean,
      Integer,
      Real},
  keywordstyle=[1]\color{red},
  morekeywords=[2]{ % keywords
    algorithm,
    and,
    annotation,
    assert,
    block,
    class,
    connector,
    constant,
    discrete,
    else,
    elseif,
    elsewhen,
    end,
    equation,
    exit,
    extends,
    external,
    false,
    final,
    flow,
    for,
    function,
    if,
    in,
    inner,
    input,
    import,
    loop,
    model,
    nondiscrete,
    not,
    or,
    outer,
    output,
    package,
    parameter,
    public,
    protected,
    record,
    redeclare,
    replaceable,
    return,
    size,
    terminate,
    then,
    true,
    type,
    when,
    while},
  keywordstyle=[2]\color{darkred},
  morekeywords=[3]{ % functions
    abs,
    acos,
    asin,
    atan,
    atan2,
    Complex,
    connect,
    conj,
    cos,
    cosh,
    cross,
    der,
    edge,
    exp,
    fromPolar,
    imag,
    noEvent,
    pre,
    sign,
    sin,
    sinh,
    sqrt,
    tan,
    tanh},
  keywordstyle=[3]\color{blue},
  morecomment=[l][\color{green}]{//}, % comments
  morecomment=[s][\color{green}]{/*}{*/}, % comments
  morestring=[b][\color{pink}]{'}, % strings
  morestring=[b][\color{pink}]{"}, % strings
}

\usepackage{tikz}
\definecolor{blue}{rgb}{.3,.5,1}
\definecolor{red}{rgb}{1,0,0}
\usetikzlibrary{shapes,arrows}
% Define block styles
\tikzstyle{decision} = [diamond, draw, fill=blue!20,
    text width=4.5em, text badly centered, node distance=3cm, inner sep=0pt]
\tikzstyle{block} = [rectangle, draw, fill=blue!20,
    text width=5em, text centered, rounded corners, minimum height=4em]
\tikzstyle{line} = [draw, -latex']
\tikzstyle{cloud} = [draw, ellipse,fill=red!20, node distance=3cm,
    minimum height=2em]
    \usetikzlibrary{calc}


\usepackage{xspace} % Define typography
\usepackage{dirtree}
\newcommand{\Dynawo}[0]{Dyna$\omega$o\xspace}


\begin{document}

\title{Dynaflow-launcher Functional Documentation}
\date\today

\maketitle
\tableofcontents

\chapter{Functional documentation}

\section[Dynaflow-launcher Overview]{Dynaflow-launcher Overview}
Dynaflow-launcher creates, for a given .iidm network file and config.json, all the input files needed for a DynaFlow simulation, runs a simulation and collects the results.
\par In particular, it generates the .dyd file containing all the modelling choices for all the system's elements and the .par file that contains all the parameters for these models.
\par The next section presents the modelling choices that are made.

\section{Modelling}

The modelling choices are based on two inputs: the configuration file which is defined by the user, and on the network description file (iidm).
The description of the behavior of the models mentionned below can be found in the \Dynawo documentation.

\subsection{Loads}

The models that are used for the loads depend on the parameter \textit{DsoVoltageLevel} in the config file (see section \ref{Dynaflow_Launcher_Configuration_Configuration_File}).
\begin{table}[h!]
\center
\begin{tabular}{ l | c }
\toprule
\textbf{{Voltage condition}} & \textbf{{Model}}\\
\midrule
 $voltage < DsoVoltageLevel$ &  fixed PQ load \\
 $voltage \geq DsoVoltageLevel$ &  DYNModelLoadRestorativeWithLimits \\
\bottomrule
\end{tabular}
\caption{Loads modelling}
\end{table}

\subsection{Generators}

The following parameters from the configuration file are used for generators modelling (see section \ref{Dynaflow_Launcher_Configuration_Configuration_File}):
\begin{itemize}
  \item The parameter \textit{InfiniteReactiveLimits} determines whether the generator models have infinite reactive power limits or PQ diagrams;
  \item The parameter \textit{ActivePowerCompensation} determines whether the generators participate in the active power balancing proportionally to PMax, P or PTarget.
\end{itemize}

The information that a generator partipates or not in the voltage regulation is given in the IIDM input file.

\par Tables \ref{tab:generators_modelling} and \ref{tab:generators_modelling_tfo} sum up the modelling choices applied by dynaflow-launcher.
The model chosen depends on the following:
\begin{itemize}
  \item if a generator is on voltage regulation or not and where the voltage is controlled;
  \item if the voltage level is above or below the $TfoVoltageLevel$ option;
  \item if the InfiniteReactiveLimits parameter is set to true or not;
  \item if the reactive capability curve (diagram) of the generator is rectangular (all the values of $Q_{min}$ are equal and all the values of $Q_{max}$ are equal).
\end{itemize}

\begin{table}[h!]
\center
\begin{tabular}{ c | c | c | c}
\toprule
\scriptsize{\textbf{{Voltage regulation}}} & \scriptsize{\textbf{{InfiniteReactiveLimits}}} & \scriptsize{\textbf{{Rectangular diagram}}} & \scriptsize{\textbf{{Model}}}\\
\midrule
\rowcolor{white}
 \scriptsize{None}  & \scriptsize{-} & \scriptsize{-} & \scriptsize{fixed PQ generator} \\
\rowcolor{gray!10}
 \scriptsize{At the connection bus} & \scriptsize{false} & \scriptsize{false} & \scriptsize{GeneratorPVDiagramPQSignalN} \\
\rowcolor{white}
 \scriptsize{At the connection bus} & \scriptsize{false} & \scriptsize{true} & \scriptsize{GeneratorPVSignalN} \\
\rowcolor{gray!10}
 \scriptsize{At the connection bus} & \scriptsize{true} & \scriptsize{-} & \scriptsize{GeneratorPVSignalN} \\
\rowcolor{white}
 \scriptsize{At a distant bus} & \scriptsize{false} & \scriptsize{false} & \scriptsize{GeneratorPVRemoteDiagramPQSignalN} \\
\rowcolor{gray!10}
 \scriptsize{At a distant bus} & \scriptsize{false} & \scriptsize{true} & \scriptsize{GeneratorPVRemoteSignalN} \\
\rowcolor{white}
 \scriptsize{At a distant bus} & \scriptsize{true} & \scriptsize{-} & \scriptsize{GeneratorPVRemoteSignalN} \\
\rowcolor{gray!10}
\scriptsize{Another generator}&  &  & \scriptsize{GeneratorPQPropDiagramPQSignalN}\\
\rowcolor{gray!10}
\scriptsize{regulates the same node} & \multirow{-2}{*}{\scriptsize{false}} & \multirow{-2}{*}{\scriptsize{false}} & \scriptsize{+ VRRemote model} \\
\rowcolor{white}
\scriptsize{Another generator}&  &  & \scriptsize{GeneratorPQPropSignalN}\\
\rowcolor{white}
\scriptsize{regulates the same node} & \multirow{-2}{*}{\scriptsize{false}} & \multirow{-2}{*}{\scriptsize{true}} & \scriptsize{+ VRRemote model} \\
\rowcolor{gray!10}
\scriptsize{Another generator}&  & & \scriptsize{GeneratorPQPropSignalN}  \\
\rowcolor{gray!10}
\scriptsize{regulates the same node} & \multirow{-2}{*}{\scriptsize{true}} & \multirow{-2}{*}{\scriptsize{-}} & \scriptsize{+ VRRemote model} \\
\bottomrule
\end{tabular}
\caption{Generators modelling when $voltageLevel \leq TfoVoltageLevel$}
\label{tab:generators_modelling}
\end{table}

\begin{table}[h!]
\center
\begin{tabular}{ c | c | c | c}
\toprule
\scriptsize{\textbf{{Voltage regulation}}} & \scriptsize{\textbf{{InfiniteReactiveLimits}}} & \scriptsize{\textbf{{Rectangular diagram}}} & \scriptsize{\textbf{{Model}}}\\
\midrule
\rowcolor{white}
 \scriptsize{None}  & \scriptsize{-} & \scriptsize{-} & \scriptsize{fixed PQ generator} \\
\rowcolor{gray!10}
 \scriptsize{At the connection bus} & \scriptsize{false} & \scriptsize{false} & \scriptsize{GeneratorPVTfoDiagramPQSignalN} \\
\rowcolor{white}
 \scriptsize{At the connection bus} & \scriptsize{false} & \scriptsize{true} & \scriptsize{GeneratorPVTfoSignalN} \\
\rowcolor{gray!10}
 \scriptsize{At the connection bus} & \scriptsize{true} & \scriptsize{-} & \scriptsize{GeneratorPVTfoSignalN} \\
\rowcolor{white}
\scriptsize{Another generator}&  &  & \\
\rowcolor{white}
\scriptsize{regulates the same node} & \multirow{-2}{*}{\scriptsize{false}} & \multirow{-2}{*}{\scriptsize{false}} & \multirow{-2}{*}{\scriptsize{GeneratorPVTfoDiagramPQSignalN}}  \\
\rowcolor{gray!10}
\scriptsize{Another generator}&  & & \\
\rowcolor{gray!10}
\scriptsize{regulates the same node} & \multirow{-2}{*}{\scriptsize{false}} & \multirow{-2}{*}{\scriptsize{true}} & \multirow{-2}{*}{\scriptsize{GeneratorPVTfoSignalN}} \\
\rowcolor{white}
\scriptsize{Another generator}&  & &  \\
\rowcolor{white}
\scriptsize{regulates the same node} & \multirow{-2}{*}{\scriptsize{true}} & \multirow{-2}{*}{\scriptsize{-}} & \multirow{-2}{*}{\scriptsize{GeneratorPVTfoSignalN}} \\
\rowcolor{gray!10}
 \scriptsize{At a distant bus} & \scriptsize{-} & \scriptsize{-} & \scriptsize{\bf{Not Supported}} \\
\bottomrule
\end{tabular}
\caption{Generators modelling when $voltageLevel > TfoVoltageLevel$}
\label{tab:generators_modelling_tfo}
\end{table}

Please refer to \ref{DFL_SVC_Rpcl_Generators} for generators belonging to a secondary voltage control area.\\

\par Concerning the active power balancing:

\begin{itemize}
  \item A generator that does not participate in the voltage regulation cannot participate in the acting power balancing;
  \item A generator with an active power target equal to 0 does not participate in the active power balancing. It can however regulate the voltage;
  \item the active power reference of the others generator define if they participate in the active power balancing.
\end{itemize}

\subsection{HVDC}

Two types of HVDC links are modelled in DynaFlow : LCCs and VSCs.

The parameter \textit{InfiniteReactiveLimits} from the configuration file determines whether the HVDC models have infinite reactive power limits
or PQ diagrams (see section \ref{Dynaflow_Launcher_Configuration_Configuration_File}).

\subsubsection{VSCs}

The conditions and models used are given below.
The information that a HVDC has an AC regulation law is given in the IIDM input file.\\

\begin{table}[ht!]
\center
\begin{tabular}{ c | c | c}
\toprule
\small{\textbf{{Voltage regulation}}}& \small{\textbf{{InfiniteReactiveLimits}}} & \small{\textbf{{Model}}} \\
\midrule
\rowcolor{gray!10}
 \small{At the connection bus} & \small{false}& \small{HvdcPVDiagramPQ} \\
\rowcolor{white}
 \small{At the connection bus} & \small{true}& \small{HvdcPV} \\
\rowcolor{gray!10}
 \small{At a distant bus} & \small{false} & \small{HvdcPQPropDiagramPQ} \\
\rowcolor{white}
 \small{At a distant bus} & \small{true} & \small{HvdcPQProp} \\
\bottomrule
\end{tabular}
\caption{VSCs modelling (not dangling, no AC emulation law)}
\end{table}

\begin{table}[ht!]
\center
\begin{tabular}{ c | c | c}
\toprule
\footnotesize{\textbf{{Voltage regulation}}}& \footnotesize{\textbf{{InfiniteReactiveLimits}}} & \small{\textbf{{Model}}} \\
\midrule
\rowcolor{gray!10}
 \footnotesize{At the connection bus} & \small{false}& \footnotesize{HvdcPVDiagramPQEmulationSet} \\
\rowcolor{white}
 \footnotesize{At the connection bus} & \small{true}& \footnotesize{HvdcPVEmulationSet} \\
\rowcolor{gray!10}
 \footnotesize{At a distant bus} & \small{false} & \footnotesize{HvdcPQPropDiagramPQEmulationSet} \\
\rowcolor{white}
 \footnotesize{At a distant bus} & \small{true} & \footnotesize{HvdcPQPropEmulationSet} \\
\bottomrule
\end{tabular}
\caption{VSCs modelling (not dangling, AC emulation law)}
\end{table}

\begin{table}[ht!]
\center
\begin{tabular}{ c | c | c }
\toprule
\footnotesize{\textbf{{Voltage regulation}}}& \footnotesize{\textbf{{InfiniteReactiveLimits}}} & \small{\textbf{{Model}}} \\
\midrule
\rowcolor{gray!10}
 \footnotesize{At the connection bus} & \small{false}& \footnotesize{HvdcPVDanglingDiagramPQ} \\
\rowcolor{white}
 \footnotesize{At the connection bus} & \small{true}& \footnotesize{HvdcPVDangling} \\
\rowcolor{gray!10}
 \footnotesize{At a distant bus} & \small{false} & \footnotesize{HvdcPQPropDanglingDiagramPQ} \\
\rowcolor{white}
 \footnotesize{At a distant bus} & \small{true} & \footnotesize{HvdcPQPropDangling} \\
\bottomrule
\end{tabular}
\caption{VSCs modelling (dangling)}
\end{table}

\subsubsection{LCCs}

The conditions and models used are given below.

\begin{table}[ht!]
\center
\begin{tabular}{ c | c | c }
\toprule
\small{\textbf{{Dangling}}}& \small{\textbf{{InfiniteReactiveLimits}}} & \small{\textbf{{Model}}} \\
\midrule
\rowcolor{white}
 \small{false} & \small{false}  & \small{HvdcPTanPhi} \\
\rowcolor{gray!10}
 \small{true} & \small{false}& \small{HvdcPTanPhiDangling} \\
\rowcolor{white}
 \small{false} & \small{true}& \small{HvdcPTanPhiDiagramPQ} \\
\rowcolor{gray!10}
 \small{true} & \small{true} & \small{HvdcPTanPhiDanglingDiagramPQ} \\
\bottomrule
\end{tabular}
\caption{LCCs modelling}
\end{table}

\subsection{Static Var Compensators}

The conditions and models used are given below.

\begin{table}[h!]
\center
\begin{tabular}{ c | c | c | c}
\toprule
& & \scriptsize{\textbf{{Custom mode}}} &  \\
\scriptsize{\multirow{-2}{*}{\textbf{{Voltage regulation}}}}& \multirow{-2}{*}{\scriptsize{\textbf{{Control law}}}}& \scriptsize{\textbf{{control law}}} & \multirow{-2}{*}{\scriptsize{\textbf{{Model}}}} \\
\midrule
\rowcolor{white}
 \scriptsize{At the connection bus} & \scriptsize{None}& \scriptsize{false}& \scriptsize{StaticVarCompensatorPV} \\
\rowcolor{gray!10}
 \scriptsize{At the connection bus} & \scriptsize{None}& \scriptsize{true}& \scriptsize{StaticVarCompensatorPVModeHandling} \\
\rowcolor{white}
 \scriptsize{At the connection bus} & \scriptsize{U + Lambda*Q}& \scriptsize{false}& \scriptsize{StaticVarCompensatorPVProp} \\
\rowcolor{gray!10}
 \scriptsize{At the connection bus} & \scriptsize{U + Lambda*Q}& \scriptsize{true}& \scriptsize{StaticVarCompensatorPVPropModeHandling} \\
\rowcolor{white}
 \scriptsize{At a distant bus} & \scriptsize{None}& \scriptsize{false}& \scriptsize{StaticVarCompensatorPVRemote} \\
\rowcolor{gray!10}
 \scriptsize{At a distant bus} & \scriptsize{None}& \scriptsize{true}& \scriptsize{StaticVarCompensatorPVRemoteModeHandling} \\
\rowcolor{white}
 \scriptsize{At a distant bus} & \scriptsize{U + Lambda*Q}& \scriptsize{false}& \scriptsize{StaticVarCompensatorPVPropRemote} \\
\rowcolor{gray!10}
 \scriptsize{At a distant bus} & \scriptsize{U + Lambda*Q}& \scriptsize{true}& \scriptsize{StaticVarCompensatorPVPropRemoteModeHandling} \\
\bottomrule
\end{tabular}
\caption{Static Var Compensators modelling}
\end{table}

\subsection{Dynamic models}
\label{DFL_Dyn_Models}

A generic way to simulate components not defined in the static model (e.g. special protection schemes) is provided in Dynaflow-launcher through the optional xml assembling and setting files.

\subsubsection{Defining an special protection scheme automaton in the assembling and setting files: general case}
\paragraph{Defining a dynamicAutomaton in the assembling and setting\\}

To define any special protection scheme automaton in Dynaflow-launcher, the assembling should contains a dynamicAutomaton. An example is given below to define a PhaseShifter.

\begin{lstlisting}[language=XML, breaklines=true, breakatwhitespace=false, columns=fullflexible]
<assembling>
...
  <dynamicAutomaton id="PhaseShifter_5_6" lib="PhaseShifterI">
    <macroConnect macroConnection="PhaseShifterToIMeasurement" id="MESURE_I_PHASE_5_6"/>
    <macroConnect macroConnection="PhaseShifterToTap" id="TAP_PHASE_5_6"/>
    <macroConnect macroConnection="PhaseShifterToAutomatonActivated" id="AUTOMATON_EXISTS_PHASE_5_6"/>
  </dynamicAutomaton>
...
</assembling>
\end{lstlisting}

The \textit{dynamicAutomaton} provides a unique id, the name of the dynamic model library
to use and a list of \textit{macroConnect} that defines the connections of the special protection scheme automaton.

The setting of this special protection scheme automaton should be given in the setting file, with the same id as the \textit{dynamicAutomaton} id:

\begin{lstlisting}[language=XML, breaklines=true, breakatwhitespace=false, columns=fullflexible]
<setting>
...
  <set id="PhaseShifter_5_6">
    <par type="DOUBLE" name="phaseShifter_sign" value="-1"/>
    <par type="DOUBLE" name="phaseShifter_t1st" value="40"/>
    ...
    <reference type="BOOL" name="phaseShifter_regulating0" origData="IIDM" origName="regulating" componentId="_BUS____5-BUS____6-1_PS"/>
    <reference type="INT" name="phaseShifter_increasePhase" origData="IIDM" origName="increasePhase" componentId="_BUS____5-BUS____6-1_PS"/>
  </set>
...
</setting>
\end{lstlisting}


Defining a \textit{dynamicAutomaton} in the assembling will end up in the creation of a blackbox model in the dyd file of the \Dynawo simulation.
The parameters used will be transposed as they are from the setting file to the parameter file:

\begin{lstlisting}[language=XML, breaklines=true, breakatwhitespace=false, columns=fullflexible]
<dyn:dynamicModelsArchitecture xmlns:dyn="http://www.rte-france.com/dynawo">
...
  <dyn:blackBoxModel id="PhaseShifter_5_6" lib="PhaseShifterI" parFile="TestIIDM_phase_shifter.par" parId="PhaseShifter_5_6"/>
...
</dyn:dynamicModelsArchitecture>
\end{lstlisting}

\paragraph{Defining the macroConnect of the dynamicAutomaton in the assembling\\}

Each \textit{macroConnect} of a \textit{dynamicAutomaton} defines one connection of the special protection scheme automaton.

The \textit{macroConnection} of a \textit{macroConnect} is similar to the \Dynawo \textit{macroConnector} from the dyd file (see \Dynawo documentation). It describes a set
of connections to apply between the special protection scheme automaton and another component of the network.

 \begin{lstlisting}[language=XML, breaklines=true, breakatwhitespace=false, columns=fullflexible]
<assembling>
...
  <macroConnection id="PhaseShifterToIMeasurement">
    <connection var1="phaseShifter_iMonitored" var2="@NAME@_i1"/>
  </macroConnection>
  <macroConnection id="PhaseShifterToTap">
    <connection var1="phaseShifter_tap" var2="@NAME@_step"/>
  </macroConnection>
  <macroConnection id="PhaseShifterToAutomatonActivated">
    <connection var1="phaseShifter_AutomatonExists" var2="@NAME@_disable_internal_tapChanger"/>
  </macroConnection>
...
</assembling>
\end{lstlisting}

The \textit{id} of a \textit{macroConnect} refers to an association provided in the assembling.

An association allows to retrieve the network component connected to the special protection scheme automaton. \\

To define a connection to a node, a bus singleAssociation is used:
 \begin{lstlisting}[language=XML, breaklines=true, breakatwhitespace=false, columns=fullflexible]
<assembling>
...
  <singleAssociation id="MyAssoc">
    <bus voltageLevel="VoltageLevelId"/>
  </singleAssociation>
...
</assembling>
\end{lstlisting}

The association above will define a connection to the first bus of the voltage level named VoltageLevelId in the IIDM. \\

To define a connection to a line, a line singleAssociation is used:

 \begin{lstlisting}[language=XML, breaklines=true, breakatwhitespace=false, columns=fullflexible]
<assembling>
...
  <singleAssociation id="MyAssoc">
    <line name="LineId"/>
  </singleAssociation>
...
</assembling>
\end{lstlisting}

The association above will define a connection to the line named LineId in the IIDM. \\

To define a connection to a transformer, a tfo singleAssociation is used:

 \begin{lstlisting}[language=XML, breaklines=true, breakatwhitespace=false, columns=fullflexible]
<assembling>
...
  <singleAssociation id="MyAssoc">
    <tfo name="TransformerId"/>
  </singleAssociation>
...
</assembling>
\end{lstlisting}

The association above will define a connection to the transformer (two windings or three windings) named TransformerId in the IIDM. \\

To define a connection to a generator, a generator singleAssociation is used:

 \begin{lstlisting}[language=XML, breaklines=true, breakatwhitespace=false, columns=fullflexible]
<assembling>
...
  <singleAssociation id="MyAssoc">
    <generator name="GeneratorId"/>
    <generator name="GeneratorId_AlternativeName"/>
  </singleAssociation>
...
</assembling>
\end{lstlisting}

The association above will define a connection to the first generator found in the IIDM based on the list. \\

To define a connection to a set of shunt compensators, a shunt multiAssociation is used:

 \begin{lstlisting}[language=XML, breaklines=true, breakatwhitespace=false, columns=fullflexible]
<assembling>
...
  <multiAssociation id="MyAssoc">
    <shunt voltageLevel="VoltageLevelId"/>
  </multiAssociation>
...
</assembling>
\end{lstlisting}

The association above will define a connection to \textbf{all} the shunts compensator from the voltage level named VoltageLevelId in the IIDM. \\

If we complete our current example with the following associations we will end up with the equivalent connections in the dyd file of the \Dynawo simulation:
 \begin{lstlisting}[language=XML, breaklines=true, breakatwhitespace=false, columns=fullflexible]
<assembling>
...
  <singleAssociation id="MESURE_I_PHASE_5_6">
    <tfo name="_BUS____5-BUS____6-1_PS"/>
  </singleAssociation>

  <singleAssociation id="TAP_PHASE_5_6">
    <tfo name="_BUS____5-BUS____6-1_PS"/>
  </singleAssociation>

  <singleAssociation id="AUTOMATON_EXISTS_PHASE_5_6">
    <tfo name="_BUS____5-BUS____6-1_PS"/>
  </singleAssociation>
...
</assembling>
\end{lstlisting}

\begin{lstlisting}[language=XML, breaklines=true, breakatwhitespace=false, columns=fullflexible]
<dyn:dynamicModelsArchitecture xmlns:dyn="http://www.rte-france.com/dynawo">
...
  <dyn:macroConnector id="PhaseShifterToAutomatonActivated">
    <dyn:connect var1="phaseShifter_AutomatonExists" var2="@NAME@_disable_internal_tapChanger"/>
  </dyn:macroConnector>

  <dyn:macroConnector id="PhaseShifterToIMeasurement">
    <dyn:connect var1="phaseShifter_iMonitored" var2="@NAME@_i1"/>
  </dyn:macroConnector>

  <dyn:macroConnector id="PhaseShifterToTap">
    <dyn:connect var1="phaseShifter_tap" var2="@NAME@_step"/>
  </dyn:macroConnector>

  <dyn:macroConnect connector="PhaseShifterToAutomatonActivated" id1="PhaseShifter_5_6" id2="NETWORK" index1="0" name2="_BUS____5-BUS____6-1_PS"/>
  <dyn:macroConnect connector="PhaseShifterToIMeasurement" id1="PhaseShifter_5_6" id2="NETWORK" index1="0" name2="_BUS____5-BUS____6-1_PS"/>
  <dyn:macroConnect connector="PhaseShifterToTap" id1="PhaseShifter_5_6" id2="NETWORK" index1="0" name2="_BUS____5-BUS____6-1_PS"/>
...
</dyn:dynamicModelsArchitecture>
\end{lstlisting}


\subsubsection{Defining a secondary voltage control in the assembling and setting files}

The secondary voltage control automaton is a specific case from the assembling and setting.

If the assembling defines a \textit{dynamicAutomaton} using the library \textbf{SecondaryVoltageControlSimp} then the model of the generators connected to
this secondary voltage control automaton will contain a reactive power control loop (Rpcl) connected to the secondary voltage control.

This secondary voltage control automaton should be connected to one bus and some generators.
An example is given below.

\begin{lstlisting}[language=XML, breaklines=true, breakatwhitespace=false, columns=fullflexible]
<assembling>
...
  <macroConnection id="SVCToUMeasurement">
    <connection var1="secondaryVoltageControl_UpPu" var2="@NAME@_Upu_value"/>
  </macroConnection>
  <macroConnection id="SVCToGenerator" index1="true">
    <connection var1="secondaryVoltageControl_level" var2="reactivePowerControlLoop_level"/>
    <connection var1="secondaryVoltageControl_limUQUp_@INDEX@_" var2="generator_limUQUp"/>
    <connection var1="secondaryVoltageControl_limUQDown_@INDEX@_" var2="generator_limUQDown"/>
  </macroConnection>

  <singleAssociation id="UMeasurement">
    <bus voltageLevel="_BUS____1_VL"/>
  </singleAssociation>

  <singleAssociation id="GEN1">
    <generator name="_GEN____1_SM"/>
  </singleAssociation>
  <singleAssociation id="GEN2">
    <generator name="_GEN____2_SM"/>
  </singleAssociation>

  <dynamicAutomaton id="SVC" lib="SecondaryVoltageControlSimp">
    <macroConnect macroConnection="SVCToUMeasurement" id="UMeasurement"/>
    <macroConnect macroConnection="SVCToGenerator" id="GEN1"/>
    <macroConnect macroConnection="SVCToGenerator" id="GEN2"/>
  </dynamicAutomaton>
...
</assembling>
\end{lstlisting}


Similarly to any \textit{dynamicAutomaton}, a set of parameters should be given in the setting file for the \textbf{SecondaryVoltageControlSimp} model.
Moreover the setting should also contain the Rpcl parameters for all generators as follows:

\begin{lstlisting}[language=XML, breaklines=true, breakatwhitespace=false, columns=fullflexible]
<setting>
...
  <set id="SVC">
    <!-- SVC parameters -->

    <!-- GEN1 Qr -->
    <par type="DOUBLE" name="secondaryVoltageControl_Qr_1_" value="280"/>
    <!-- GEN2 Qr -->
    <par type="DOUBLE" name="secondaryVoltageControl_Qr_2_" value="280"/>

    <par type="DOUBLE" name="secondaryVoltageControl_Alpha" value="0.0308641975"/>
    <par type="DOUBLE" name="secondaryVoltageControl_Beta" value="1.2345679"/>
  </set>

  <set id="GEN1">
    <!-- GEN1 parameters -->
    <par type="DOUBLE" name="reactivePowerControlLoop_DerURefMaxPu" value="0.0002"/>
    <par type="DOUBLE" name="reactivePowerControlLoop_QrPu" value="1.4"/>
    <par type="DOUBLE" name="reactivePowerControlLoop_TiQ" value="150"/>
  </set>

  <set id="GEN2">
    <!-- GEN2 parameters -->
    <par type="DOUBLE" name="reactivePowerControlLoop_DerURefMaxPu" value="0.0002"/>
    <par type="DOUBLE" name="reactivePowerControlLoop_QrPu" value="1.4"/>
    <par type="DOUBLE" name="reactivePowerControlLoop_TiQ" value="150"/>
  </set>
...
</setting>
\end{lstlisting}


To define a generator using a reactive power control loop 2 (Rpcl2) then the \textit{multipleAssociation}
of the generator should be referenced in the \textit{ReactivePowerControlLoop2} property list in the assembling:


\begin{lstlisting}[language=XML, breaklines=true, breakatwhitespace=false, columns=fullflexible]
<assembling>
...
  <property id="ReactivePowerControlLoop2">
    <device id="GEN2"/>
    <device id="..."/>
    ...
  </property>
...
</assembling>
\end{lstlisting}

In this case the settings of the generator should be adapted to the Rpcl2:

\begin{lstlisting}[language=XML, breaklines=true, breakatwhitespace=false, columns=fullflexible]
<setting>
...
  <set id="GEN2">
    <!-- GEN2 parameters -->
    <par name="reactivePowerControlLoop_CqMaxPu" type="DOUBLE" value="15"/>
    <par name="reactivePowerControlLoop_DeltaURefMaxPu" type="DOUBLE" value="0.002"/>
    <par name="reactivePowerControlLoop_QrPu" type="DOUBLE" value="1"/>
    <par name="reactivePowerControlLoop_Tech" type="DOUBLE" value="10"/>
    <par name="reactivePowerControlLoop_Ti" type="DOUBLE" value="60"/>
  </set>
...
</setting>
\end{lstlisting}

\label{DFL_SVC_Rpcl_Generators}

The models used for generators belonging to a secondary voltage control area are given in the table \ref{tab:generators_modelling_rpcl} and \ref{tab:generators_modelling_tfo_rpcl}.\\

\textbf{Generators with the following properties will be automatically removed from the secondary voltage control area:}
\begin{itemize}
  \item Generators regulating at a distant node
  \item Generators regulating a bus already regulated by another generator and with $voltageLevel \leq TfoVoltageLevel$
\end{itemize}


Simply replace \textit{'Rpcl'} by \textit{'Rpcl2'} in the model name when the generator uses a reactive power control loop 2.\\

\begin{table}[h!]
\center
\begin{tabular}{ c | c | c | c}
\toprule
\scriptsize{\textbf{{Voltage regulation}}} & \scriptsize{\textbf{{InfiniteReactiveLimits}}} & \scriptsize{\textbf{{Rectangular diagram}}}  & \scriptsize{\textbf{{Model}}}\\
\midrule
\rowcolor{white}
 \scriptsize{At the connection bus} & \scriptsize{false} & \scriptsize{false} & \scriptsize{GeneratorPVDiagramPQRpclSignalN} \\
\rowcolor{gray!10}
 \scriptsize{At the connection bus} & \scriptsize{false} & \scriptsize{true} & \scriptsize{GeneratorPVRpclSignalN} \\
\rowcolor{white}
 \scriptsize{At the connection bus} & \scriptsize{true} & \scriptsize{-} & \scriptsize{GeneratorPVRpclSignalN} \\
\bottomrule
\end{tabular}
\caption{Generators modelling when $voltageLevel \leq TfoVoltageLevel$ and with a reactive power control loop}
\label{tab:generators_modelling_rpcl}
\end{table}

\begin{table}[h!]
\center
\begin{tabular}{ c | c | c | c}
\toprule
\scriptsize{\textbf{{Voltage regulation}}} & \scriptsize{\textbf{{InfiniteReactiveLimits}}} & \scriptsize{\textbf{{Rectangular diagram}}}  & \scriptsize{\textbf{{Model}}}\\
\midrule
\rowcolor{white}
 \scriptsize{At the connection bus} & \scriptsize{false} & \scriptsize{false} & \scriptsize{GeneratorPVTfoDiagramPQRpclSignalN} \\
\rowcolor{gray!10}
 \scriptsize{At the connection bus} & \scriptsize{false} & \scriptsize{true} & \scriptsize{GeneratorPVTfoRpclSignalN} \\
\rowcolor{white}
 \scriptsize{At the connection bus} & \scriptsize{true} & \scriptsize{-} & \scriptsize{GeneratorPVTfoRpclSignalN} \\
\bottomrule
\end{tabular}
\caption{Generators modelling when $voltageLevel > TfoVoltageLevel$ and with a reactive power control loop}
\label{tab:generators_modelling_tfo_rpcl}
\end{table}

\section{Solver}

The solver used is the Simplified Solver (SolverSIM) from \Dynawo.
The Maximum time step value is by default 10s. This value can be modified in the configuration file (see section \ref{Dynaflow_Launcher_Configuration_Configuration_File}).
Notice that the maximum step value should be reduced only when the simulation is including
Special Protection Scheme automatons with a timescale smaller than 10 seconds: this would allow one to properly
take into account the SPS behaviour, avoiding any artificial synchronization effect. When modifying the
solver maximum time step, its value must be included in $ 1.0  < TimeStep \leq 10.0 $ seconds.

The full configuration (with default values) used is the following:

\lstinputlisting[language=XML,title=\Dynawo Simplified Solver configuration]{../resources/syntaxExample/solver.par}

\section{Assembling and Setting files}

The following parameters from the configuration file are used to add additional models and their settings to the system's elements defined in the given .iidm network file (see section \ref{Dynaflow_Launcher_Configuration_Configuration_File}):
\begin{itemize}
  \item The parameter \textit{AssemblingPath} is the path to the assembling file;
  \item The parameter \textit{SettingPath} is the path to the setting file.
\end{itemize}

These two files use a syntax close to the one used for the .dyd and .par files. It's available here etc/xsd/.

\subsection{Assembling file}

The assembling file is a \textbf{.xml} file containing connections between system's elements and models. An example is given below:

\lstinputlisting[language=XML,title=\Dynawo Assembling file example]{../resources/syntaxExample/assembling.xml}

\subsubsection{Use a model}

A dynamicAutomaton statement is required to define a model. The user must also specify an id and the model librairy:

\begin{lstlisting}[language=XML, morekeywords={dynamicAutomaton}]
<dynamicAutomaton id="MODELE_1_VL4" lib="DYNModel1">
  ...
</dynamicAutomaton>
\end{lstlisting}

\subsubsection{Defines sets of connections}

The connection between two different models often needs more than one ``connect'' statement. For example, in order to connect DYNModel1 to the Network we need to connect three variables: shunt\_state, shunt\_isCapacitor and shunt\_isAvailable.
In order not to write the three connections for each DYNModel1 model of the network, we can define a macro connection:
\begin{lstlisting}[language=XML, morekeywords={macroConnection}]
<macroConnection id="ToControlledShunts">
  <connection var1="shunt_state_@INDEX@" var2="@NAME@_state"/>
  <connection var1="shunt_isCapacitor_@INDEX@" var2="@NAME@_isCapacitor"/>
  <connection var1="shunt_isAvailable_@INDEX@" var2="@NAME@_isAvailable"/>
</macroConnection>
\end{lstlisting}

\subsubsection{Connect a model variable to an other model}

In order to connect a variable of a model defined in dynamicAutomaton tags to an other model variable, the user can use a macroConnect. It takes:
\begin{itemize}
  \item A macro connection, to link a current model variable to an other model variable;
  \item An id referring to a system's element (in this cas, the other model will be the Network) or an other model name (see next section \ref{Node_Or_Model_Reference_On_Macro_Connect})
\end{itemize}

\begin{lstlisting}[language=XML, morekeywords={macroConnect}]
<dynamicAutomaton id="MODELE_1_VL4" lib="DYNModel1">
  <macroConnect macroConnection="ToUMeasurement" id="MESURE_MODELE_1_VL4"/>
  <macroConnect macroConnection="ToControlledShunts" id="SHUNTS_MODELE_1_VL4"/>
</dynamicAutomaton>
\end{lstlisting}

\subsubsection{How to reference system's element or an other model in a macroConnect statement ?}
\label{Node_Or_Model_Reference_On_Macro_Connect}

To refer to a bus, a line, a shunt or a tfo of the system, the user can use a singleAssociation statement. For example, in order to connect MODELE\_1\_VL4 variables to the variables of
the bus with voltage level VLP6, we can define a single association:

\begin{lstlisting}[language=XML, morekeywords={singleAssociation}]
<singleAssociation id="MESURE_MODELE_1_VL4">
  <bus voltageLevel="VLP6" />
</singleAssociation>
\end{lstlisting}

To refer to all the shunts associated to a voltage level of the system, the user can use a multipleAssociation statement:

\begin{lstlisting}[language=XML, morekeywords={multipleAssociation}]
<multipleAssociation id="SHUNTS_MODELE_1_VL4">
  <shunt voltageLevel="VL4"/>
</multipleAssociation>
\end{lstlisting}

To refer to an other model, the user cas use a modelAssociation statement:

\begin{lstlisting}[language=XML, morekeywords={modelAssociation}]
<modelAssociation id="MY_MODEL_ASSOCIATION_ID">
  <model id="MY_MODEL_ID" lib="MY_MODEL_LIB"/>
</modelAssociation>
\end{lstlisting}

\subsection{Setting file}

The setting file is a .xml file containing the needed parameters of the models referenced in the assembling file. The syntax is the same as for .par files.

\end{document}

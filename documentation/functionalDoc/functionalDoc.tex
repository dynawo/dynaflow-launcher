%% Except where otherwise noted, content in this documentation is Copyright (c)
%% 2022, RTE (http://www.rte-france.com) and licensed under a
%% CC-BY-4.0 (https://creativecommons.org/licenses/by/4.0/)
%% license. All rights reserved.

\documentclass[a4paper, 12pt]{report}

% Latex setup
\input{../latex_setup.tex}

\begin{document}

\title{Dynaflow-launcher Functional Documentation}
\date\today

\maketitle
\tableofcontents

\chapter{Functional documentation}

\section[Dynaflow-launcher Overview]{Dynaflow-launcher Overview}
Dynaflow-launcher creates, for a given .iidm network file and config.json, all the input files needed for a DynaFlow simulation, runs a simulation and collects the results.
\par In particular, it generates the .dyd file containing all the modelling choices for all the system's elements and the .par file that contains all the parameters for these models.
\par The next section presents the modelling choices that are made.

\section{Modelling}

The modelling choices are based on two inputs: the configuration file which is defined by the user, and on the network description file (iidm).
The description of the behavior of the models mentionned below can be found in the \Dynawo documentation.

\subsection{Loads}

The models that are used for the loads depend on the parameter \textit{DsoVoltageLevel} in the config file (see section \ref{Dynaflow_Launcher_Configuration_Configuration_File}).
\begin{table}[h!]
\center
\begin{tabular}{ l | c }
\toprule
\textbf{{Voltage condition}} & \textbf{{Model}}\\
\midrule
 $voltage < DsoVoltageLevel$ &  fixed PQ load \\
 $voltage \geq DsoVoltageLevel$ &  DYNModelLoadRestorativeWithLimits \\
\bottomrule
\end{tabular}
\caption{Loads modelling}
\end{table}

\subsection{Generators}

The following parameters from the configuration file are used for generators modelling (see section \ref{Dynaflow_Launcher_Configuration_Configuration_File}):
\begin{itemize}
  \item The parameter \textit{InfiniteReactiveLimits} determines whether the generator models have infinite reactive power limits or PQ diagrams;
  \item The parameter \textit{ActivePowerCompensation} determines whether the generators participate in the active power balancing proportionally to PMax, P or PTarget.
\end{itemize}

The information that a generator partipates or not in the voltage regulation is given in the IIDM input file.

\par Tables \ref{tab:generators_modelling} and \ref{tab:generators_modelling_tfo} sum up the modelling choices applied by dynaflow-launcher.
The model chosen depends on the following:
\begin{itemize}
  \item if a generator is on voltage regulation or not and where the voltage is controlled;
  \item if the voltage level is above or below the $TfoVoltageLevel$ option;
  \item if the InfiniteReactiveLimits parameter is set to true or not;
  \item if the reactive capability curve (diagram) of the generator is rectangular (all the values of $Q_{min}$ are equal and all the values of $Q_{max}$ are equal).
\end{itemize}

\begin{table}[h!]
\center
\begin{tabular}{ c | c | c | c}
\toprule
\scriptsize{\textbf{{Voltage regulation}}} & \scriptsize{\textbf{{InfiniteReactiveLimits}}} & \scriptsize{\textbf{{Rectangular diagram}}} & \scriptsize{\textbf{{Model}}}\\
\midrule
\rowcolor{white}
 \scriptsize{None}  & \scriptsize{-} & \scriptsize{-} & \scriptsize{fixed PQ generator} \\
\rowcolor{gray!10}
 \scriptsize{At the connection bus} & \scriptsize{false} & \scriptsize{false} & \scriptsize{GeneratorPVDiagramPQSignalN} \\
\rowcolor{white}
 \scriptsize{At the connection bus} & \scriptsize{false} & \scriptsize{true} & \scriptsize{GeneratorPVSignalN} \\
\rowcolor{gray!10}
 \scriptsize{At the connection bus} & \scriptsize{true} & \scriptsize{-} & \scriptsize{GeneratorPVSignalN} \\
\rowcolor{white}
 \scriptsize{At a distant bus} & \scriptsize{false} & \scriptsize{false} & \scriptsize{GeneratorPVRemoteDiagramPQSignalN} \\
\rowcolor{gray!10}
 \scriptsize{At a distant bus} & \scriptsize{false} & \scriptsize{true} & \scriptsize{GeneratorPVRemoteSignalN} \\
\rowcolor{white}
 \scriptsize{At a distant bus} & \scriptsize{true} & \scriptsize{-} & \scriptsize{GeneratorPVRemoteSignalN} \\
\rowcolor{gray!10}
\scriptsize{Another generator}&  &  & \scriptsize{GeneratorPQPropDiagramPQSignalN}\\
\rowcolor{gray!10}
\scriptsize{regulates the same node} & \multirow{-2}{*}{\scriptsize{false}} & \multirow{-2}{*}{\scriptsize{false}} & \scriptsize{+ VRRemote model} \\
\rowcolor{white}
\scriptsize{Another generator}&  &  & \scriptsize{GeneratorPQPropSignalN}\\
\rowcolor{white}
\scriptsize{regulates the same node} & \multirow{-2}{*}{\scriptsize{false}} & \multirow{-2}{*}{\scriptsize{true}} & \scriptsize{+ VRRemote model} \\
\rowcolor{gray!10}
\scriptsize{Another generator}&  & & \scriptsize{GeneratorPQPropSignalN}  \\
\rowcolor{gray!10}
\scriptsize{regulates the same node} & \multirow{-2}{*}{\scriptsize{true}} & \multirow{-2}{*}{\scriptsize{-}} & \scriptsize{+ VRRemote model} \\
\bottomrule
\end{tabular}
\caption{Generators modelling when $voltageLevel \leq TfoVoltageLevel$}
\label{tab:generators_modelling}
\end{table}

\begin{table}[h!]
\center
\begin{tabular}{ c | c | c | c}
\toprule
\scriptsize{\textbf{{Voltage regulation}}} & \scriptsize{\textbf{{InfiniteReactiveLimits}}} & \scriptsize{\textbf{{Rectangular diagram}}} & \scriptsize{\textbf{{Model}}}\\
\midrule
\rowcolor{white}
 \scriptsize{None}  & \scriptsize{-} & \scriptsize{-} & \scriptsize{fixed PQ generator} \\
\rowcolor{gray!10}
 \scriptsize{At the connection bus} & \scriptsize{false} & \scriptsize{false} & \scriptsize{GeneratorPVTfoDiagramPQSignalN} \\
\rowcolor{white}
 \scriptsize{At the connection bus} & \scriptsize{false} & \scriptsize{true} & \scriptsize{GeneratorPVTfoSignalN} \\
\rowcolor{gray!10}
 \scriptsize{At the connection bus} & \scriptsize{true} & \scriptsize{-} & \scriptsize{GeneratorPVTfoSignalN} \\
\rowcolor{white}
\scriptsize{Another generator}&  &  & \\
\rowcolor{white}
\scriptsize{regulates the same node} & \multirow{-2}{*}{\scriptsize{false}} & \multirow{-2}{*}{\scriptsize{false}} & \multirow{-2}{*}{\scriptsize{GeneratorPVTfoDiagramPQSignalN}}  \\
\rowcolor{gray!10}
\scriptsize{Another generator}&  & & \\
\rowcolor{gray!10}
\scriptsize{regulates the same node} & \multirow{-2}{*}{\scriptsize{false}} & \multirow{-2}{*}{\scriptsize{true}} & \multirow{-2}{*}{\scriptsize{GeneratorPVTfoSignalN}} \\
\rowcolor{white}
\scriptsize{Another generator}&  & &  \\
\rowcolor{white}
\scriptsize{regulates the same node} & \multirow{-2}{*}{\scriptsize{true}} & \multirow{-2}{*}{\scriptsize{-}} & \multirow{-2}{*}{\scriptsize{GeneratorPVTfoSignalN}} \\
\rowcolor{gray!10}
 \scriptsize{At a distant bus} & \scriptsize{-} & \scriptsize{-} & \scriptsize{\bf{Not Supported}} \\
\bottomrule
\end{tabular}
\caption{Generators modelling when $voltageLevel > TfoVoltageLevel$}
\label{tab:generators_modelling_tfo}
\end{table}

\par Concerning the active power balancing:

\begin{itemize}
  \item A generator that does not participate in the voltage regulation cannot participate in the acting power balancing;
  \item A generator with an active power target equal to 0 does not participate in the active power balancing. It can however regulate the voltage;
  \item the active power reference of the others generator define if they participate in the active power balancing.
\end{itemize}

\subsection{HVDC}

Two types of HVDC links are modelled in DynaFlow : LCCs and VSCs.

The parameter \textit{InfiniteReactiveLimits} from the configuration file determines whether the HVDC models have infinite reactive power limits
or PQ diagrams (see section \ref{Dynaflow_Launcher_Configuration_Configuration_File}).

\subsubsection{VSCs}

The conditions and models used are given below.
The information that a HVDC has an AC regulation law is given in the IIDM input file.\\

\begin{table}[ht!]
\center
\begin{tabular}{ c | c | c}
\toprule
\small{\textbf{{Voltage regulation}}}& \small{\textbf{{InfiniteReactiveLimits}}} & \small{\textbf{{Model}}} \\
\midrule
\rowcolor{gray!10}
 \small{At the connection bus} & \small{false}& \small{HvdcPVDiagramPQ} \\
\rowcolor{white}
 \small{At the connection bus} & \small{true}& \small{HvdcPV} \\
\rowcolor{gray!10}
 \small{At a distant bus} & \small{false} & \small{HvdcPQPropDiagramPQ} \\
\rowcolor{white}
 \small{At a distant bus} & \small{true} & \small{HvdcPQProp} \\
\bottomrule
\end{tabular}
\caption{VSCs modelling (not dangling, no AC emulation law)}
\end{table}

\begin{table}[ht!]
\center
\begin{tabular}{ c | c | c}
\toprule
\footnotesize{\textbf{{Voltage regulation}}}& \footnotesize{\textbf{{InfiniteReactiveLimits}}} & \small{\textbf{{Model}}} \\
\midrule
\rowcolor{gray!10}
 \footnotesize{At the connection bus} & \small{false}& \footnotesize{HvdcPVDiagramPQEmulationSet} \\
\rowcolor{white}
 \footnotesize{At the connection bus} & \small{true}& \footnotesize{HvdcPVEmulationSet} \\
\rowcolor{gray!10}
 \footnotesize{At a distant bus} & \small{false} & \footnotesize{HvdcPQPropDiagramPQEmulationSet} \\
\rowcolor{white}
 \footnotesize{At a distant bus} & \small{true} & \footnotesize{HvdcPQPropEmulationSet} \\
\bottomrule
\end{tabular}
\caption{VSCs modelling (not dangling, AC emulation law)}
\end{table}

\begin{table}[ht!]
\center
\begin{tabular}{ c | c | c }
\toprule
\footnotesize{\textbf{{Voltage regulation}}}& \footnotesize{\textbf{{InfiniteReactiveLimits}}} & \small{\textbf{{Model}}} \\
\midrule
\rowcolor{gray!10}
 \footnotesize{At the connection bus} & \small{false}& \footnotesize{HvdcPVDanglingDiagramPQ} \\
\rowcolor{white}
 \footnotesize{At the connection bus} & \small{true}& \footnotesize{HvdcPVDangling} \\
\rowcolor{gray!10}
 \footnotesize{At a distant bus} & \small{false} & \footnotesize{HvdcPQPropDanglingDiagramPQ} \\
\rowcolor{white}
 \footnotesize{At a distant bus} & \small{true} & \footnotesize{HvdcPQPropDangling} \\
\bottomrule
\end{tabular}
\caption{VSCs modelling (dangling)}
\end{table}

\subsubsection{LCCs}

The conditions and models used are given below.

\begin{table}[ht!]
\center
\begin{tabular}{ c | c | c }
\toprule
\small{\textbf{{Dangling}}}& \small{\textbf{{InfiniteReactiveLimits}}} & \small{\textbf{{Model}}} \\
\midrule
\rowcolor{white}
 \small{false} & \small{false}  & \small{HvdcPTanPhi} \\
\rowcolor{gray!10}
 \small{true} & \small{false}& \small{HvdcPTanPhiDangling} \\
\rowcolor{white}
 \small{false} & \small{true}& \small{HvdcPTanPhiDiagramPQ} \\
\rowcolor{gray!10}
 \small{true} & \small{true} & \small{HvdcPTanPhiDanglingDiagramPQ} \\
\bottomrule
\end{tabular}
\caption{LCCs modelling}
\end{table}

\subsection{Static Var Compensators}

The conditions and models used are given below.

\begin{table}[h!]
\center
\begin{tabular}{ c | c | c | c}
\toprule
& & \scriptsize{\textbf{{Custom mode}}} &  \\
\scriptsize{\multirow{-2}{*}{\textbf{{Voltage regulation}}}}& \multirow{-2}{*}{\scriptsize{\textbf{{Control law}}}}& \scriptsize{\textbf{{control law}}} & \multirow{-2}{*}{\scriptsize{\textbf{{Model}}}} \\
\midrule
\rowcolor{white}
 \scriptsize{At the connection bus} & \scriptsize{None}& \scriptsize{false}& \scriptsize{StaticVarCompensatorPV} \\
\rowcolor{gray!10}
 \scriptsize{At the connection bus} & \scriptsize{None}& \scriptsize{true}& \scriptsize{StaticVarCompensatorPVModeHandling} \\
\rowcolor{white}
 \scriptsize{At the connection bus} & \scriptsize{U + Lambda*Q}& \scriptsize{false}& \scriptsize{StaticVarCompensatorPVProp} \\
\rowcolor{gray!10}
 \scriptsize{At the connection bus} & \scriptsize{U + Lambda*Q}& \scriptsize{true}& \scriptsize{StaticVarCompensatorPVPropModeHandling} \\
\rowcolor{white}
 \scriptsize{At a distant bus} & \scriptsize{None}& \scriptsize{false}& \scriptsize{StaticVarCompensatorPVRemote} \\
\rowcolor{gray!10}
 \scriptsize{At a distant bus} & \scriptsize{None}& \scriptsize{true}& \scriptsize{StaticVarCompensatorPVRemoteModeHandling} \\
\rowcolor{white}
 \scriptsize{At a distant bus} & \scriptsize{U + Lambda*Q}& \scriptsize{false}& \scriptsize{StaticVarCompensatorPVPropRemote} \\
\rowcolor{gray!10}
 \scriptsize{At a distant bus} & \scriptsize{U + Lambda*Q}& \scriptsize{true}& \scriptsize{StaticVarCompensatorPVPropRemoteModeHandling} \\
\bottomrule
\end{tabular}
\caption{Static Var Compensators modelling}
\end{table}


\section{Solver}

The solver used is the Simplified Solver (SolverSIM) from \Dynawo.
The Maximum time step value is by default 10s. This value can be modified in the configuration file (see section \ref{Dynaflow_Launcher_Configuration_Configuration_File}).
Notice that the maximum step value should be reduced only when the simulation is including
Special Protection Scheme automatons with a timescale smaller than 10 seconds: this would allow one to properly
take into account the SPS behaviour, avoiding any artificial synchronization effect. When modifying the
solver maximum time step, its value must be included in $ 1.0  < TimeStep \leq 10.0 $ seconds.

The full configuration (with default values) used is the following:

\lstinputlisting[language=XML,title=\Dynawo Simplified Solver configuration]{../resources/syntaxExample/solver.par}

\end{document}

%% Except where otherwise noted, content in this documentation is Copyright (c)
%% 2022, RTE (http://www.rte-france.com) and licensed under a
%% CC-BY-4.0 (https://creativecommons.org/licenses/by/4.0/)
%% license. All rights reserved.

\documentclass[a4paper, 12pt]{report}

% Latex setup
%% Except where otherwise noted, content in this documentation is Copyright (c)
%% 2022, RTE (http://www.rte-france.com) and licensed under a
%% CC-BY-4.0 (https://creativecommons.org/licenses/by/4.0/)
%% license. All rights reserved.

% Latin Modern fam­ily of fonts
\usepackage{lmodern}

\usepackage[english]{babel}

% specify encoding
\usepackage[utf8]{inputenc} % input
\usepackage[T1]{fontenc} % output
\usepackage[table]{xcolor}

% Document structure setup
\usepackage{titlesec} % To change chapter format
\setcounter{tocdepth}{3} % Add subsubsection in Content
\setcounter{secnumdepth}{3} % Add numbering for subsubsection
\setlength{\parindent}{0pt} % No paragraph indentation

% Change title format for chapter
\titleformat{\chapter}{\Huge\bf}{\thechapter}{20pt}{\Huge\bf}

% To add links on page number in Content and hide red rectangle on links
\usepackage[hidelinks, linktoc=all]{hyperref}
\usepackage[nottoc]{tocbibind}  % To add biblio in table of content
\usepackage{textcomp} % For single quote
\usepackage{url} % Allow linebreaks in \url command
\usepackage{listings} % To add code samples

% Default listings parameters
\lstset
{
  aboveskip={1\baselineskip}, % a bit of space above
  backgroundcolor=\color{shadecolor}, % choose the background color
  basicstyle={\ttfamily\footnotesize}, % use font and smaller size \small \footnotesize
  breakatwhitespace=true, % sets if automatic breaks should only happen at whitespace
  breaklines=true, % sets automatic line breaking
  columns=fixed, % nice spacing -> fixed / flexible
  mathescape=false, % escape to latex false
  numbers=left, % where to put the line-numbers
  numberstyle=\tiny\color{gray}, % the style that is used for the line-numbers
  showstringspaces=false, % do not emphasize spaces in strings
  tabsize=4, % number of spaces of a TAB
  texcl=false, % activates or deactivates LaTeX comment lines
  upquote=true % upright quotes
}

% Avoid numbering starting at each chapter for figures
\usepackage{chngcntr}
\counterwithout{figure}{chapter}

\usepackage{tikz} % macro pack­age for cre­at­ing graph­ics
\usepackage{pgfplots} % draws func­tion plots (based on pgf/tikz)
\pgfplotsset{enlarge x limits=false, xlabel={\begin{small}$time$ (s)\end{small}}, height=0.6\textwidth, width=1\textwidth,
yticklabel style={text width={width("$-0.6$")},align=right},
/pgf/number format/precision=4}
\pgfplotstableset{col sep=semicolon}

\usepackage{algorithm} % Add algorithms
\usepackage[noend]{algpseudocode} %  all end ... lines are omitted in algos

\usepackage{amsmath} % Add math­e­mat­i­cal fea­tures
\usepackage{schemabloc} % Add block diagram library (french one)

\usepackage{adjustbox} % Add box for flowchart

\usepackage{booktabs} % for toprule and midrule in tables

\usepackage{tabularx}
\usepackage{multirow}

\usepackage[nolist]{acronym} % don’t write the list of acronyms.
% Acronyms list
\begin{acronym}
\acro{BDF}{Backward Differentiation Formula}
\acro{BE}{Backward Euler}
\acro{DAE}{Differential Algebraic Equations}
\acro{IDA}{Implicit Differential-Algebraic solver}
\acro{LLNL}{Lawrence Livermore National Lab}
\acro{KINSOL}{Krylov Inexact Newton SOLver}
\acro{NR}{Newton-Raphson}
\acro{PLL}{Phase-Locked Loop}
\acro{SVC}{Static Var Compensator}
\acro{SUNDIALS}{SUite of Nonlinear and DIfferential/ALgebraic equation Solvers}
\end{acronym}

% Syntax highlight
%% Except where otherwise noted, content in this documentation is Copyright (c)
%% 2022, RTE (http://www.rte-france.com) and licensed under a
%% CC-BY-4.0 (https://creativecommons.org/licenses/by/4.0/)
%% license. All rights reserved.

\usepackage{color}

\definecolor{blue}{rgb}{0,0,1}
\definecolor{lightblue}{rgb}{.3,.5,1}
\definecolor{darkblue}{rgb}{0,0,.4}
\definecolor{red}{rgb}{1,0,0}
\definecolor{darkred}{rgb}{.56,0,0}
\definecolor{pink}{rgb}{.933,0,.933}
\definecolor{purple}{rgb}{0.58,0,0.82}
\definecolor{green}{rgb}{0.133,0.545,0.133}
\definecolor{darkgreen}{rgb}{0,.4,0}
\definecolor{gray}{rgb}{.3,.3,.3}
\definecolor{darkgray}{rgb}{.2,.2,.2}
\definecolor{shadecolor}{gray}{0.925}

% **********************************************************************************
% Syntax : Bash (bash)
% **********************************************************************************

\lstdefinelanguage{bash}
{
  keywordstyle=\color{blue},
  morekeywords={
    cd,
    export,
    source},
  numbers=none,
  deletekeywords={jobs}
}

% **********************************************************************************
% Syntax : XML
% **********************************************************************************

\lstdefinelanguage{XML}
{
  morestring=[s][\color{purple}]{"}{"},
  morecomment=[s][\color{green}]{<?}{?>},
  morecomment=[s][\color{green}]{<!--}{-->},
  stringstyle=\color{black},
  identifierstyle=\color{blue},
  keywordstyle=\color{red},
  morekeywords={
    xmlns,
    xsi,
    noNamespaceSchemaLocation,
    type,
    source,
    target,
    version,
    tool,
    transRef,
    roleRef,
    objective,
    eventually}
}


% **********************************************************************************
% Syntax : JSON
% **********************************************************************************

\colorlet{punct}{red!60!black}
\definecolor{background}{HTML}{EEEEEE}
\definecolor{delim}{RGB}{20,105,176}
\colorlet{numb}{magenta!60!black}
\lstdefinelanguage{json}{
    basicstyle=\normalfont\ttfamily,
    numbers=left,
    numberstyle=\scriptsize,
    stepnumber=1,
    numbersep=8pt,
    showstringspaces=false,
    breaklines=true,
    frame=lines,
    backgroundcolor=\color{background},
    literate=
     *{0}{{{\color{numb}0}}}{1}
      {1}{{{\color{numb}1}}}{1}
      {2}{{{\color{numb}2}}}{1}
      {3}{{{\color{numb}3}}}{1}
      {4}{{{\color{numb}4}}}{1}
      {5}{{{\color{numb}5}}}{1}
      {6}{{{\color{numb}6}}}{1}
      {7}{{{\color{numb}7}}}{1}
      {8}{{{\color{numb}8}}}{1}
      {9}{{{\color{numb}9}}}{1}
      {:}{{{\color{punct}{:}}}}{1}
      {,}{{{\color{punct}{,}}}}{1}
      {\{}{{{\color{delim}{\{}}}}{1}
      {\}}{{{\color{delim}{\}}}}}{1}
      {[}{{{\color{delim}{[}}}}{1}
      {]}{{{\color{delim}{]}}}}{1},
}

% **********************************************************************************
% Syntax : Modelica (modelica)
% **********************************************************************************
\lstdefinelanguage{Modelica}{
  alsoletter={...},
  morekeywords=[1]{ % types
      Boolean,
      Integer,
      Real},
  keywordstyle=[1]\color{red},
  morekeywords=[2]{ % keywords
    algorithm,
    and,
    annotation,
    assert,
    block,
    class,
    connector,
    constant,
    discrete,
    else,
    elseif,
    elsewhen,
    end,
    equation,
    exit,
    extends,
    external,
    false,
    final,
    flow,
    for,
    function,
    if,
    in,
    inner,
    input,
    import,
    loop,
    model,
    nondiscrete,
    not,
    or,
    outer,
    output,
    package,
    parameter,
    public,
    protected,
    record,
    redeclare,
    replaceable,
    return,
    size,
    terminate,
    then,
    true,
    type,
    when,
    while},
  keywordstyle=[2]\color{darkred},
  morekeywords=[3]{ % functions
    abs,
    acos,
    asin,
    atan,
    atan2,
    Complex,
    connect,
    conj,
    cos,
    cosh,
    cross,
    der,
    edge,
    exp,
    fromPolar,
    imag,
    noEvent,
    pre,
    sign,
    sin,
    sinh,
    sqrt,
    tan,
    tanh},
  keywordstyle=[3]\color{blue},
  morecomment=[l][\color{green}]{//}, % comments
  morecomment=[s][\color{green}]{/*}{*/}, % comments
  morestring=[b][\color{pink}]{'}, % strings
  morestring=[b][\color{pink}]{"}, % strings
}

\usepackage{tikz}
\definecolor{blue}{rgb}{.3,.5,1}
\definecolor{red}{rgb}{1,0,0}
\usetikzlibrary{shapes,arrows}
% Define block styles
\tikzstyle{decision} = [diamond, draw, fill=blue!20,
    text width=4.5em, text badly centered, node distance=3cm, inner sep=0pt]
\tikzstyle{block} = [rectangle, draw, fill=blue!20,
    text width=5em, text centered, rounded corners, minimum height=4em]
\tikzstyle{line} = [draw, -latex']
\tikzstyle{cloud} = [draw, ellipse,fill=red!20, node distance=3cm,
    minimum height=2em]
    \usetikzlibrary{calc}


\usepackage{xspace} % Define typography
\usepackage{dirtree}
\newcommand{\Dynawo}[0]{Dyna$\omega$o\xspace}


\begin{document}

\title{Dynaflow-launcher Functional Documentation}
\date\today

\maketitle
\tableofcontents

\chapter{Functional documentation}

\section[Dynaflow-launcher Overview]{Dynaflow-launcher Overview}
Dynaflow-launcher creates, for a given .iidm network file and config.json, all the input files needed for a DynaFlow simulation, runs a simulation and collects the results.
\par In particular, it generates the .dyd file containing all the modelling choices for all the system's elements and the .par file that contains all the parameters for these models.
\par The next section presents the modelling choices that are made.

\section{Modelling}

The modelling choices are based on two inputs: the configuration file which is defined by the user, and on the network description file (iidm).
The description of the behavior of the models mentionned below can be found in the \Dynawo documentation.

\subsection{Loads}

The models that are used for the loads depend on the parameter \textit{DsoVoltageLevel} in the config file (see section \ref{Dynaflow_Launcher_Configuration_Configuration_File}).
\begin{table}[h!]
\center
\begin{tabular}{ l | c }
\toprule
\textbf{{Voltage condition}} & \textbf{{Model}}\\
\midrule
 $voltage < DsoVoltageLevel$ &  fixed PQ load \\
 $voltage \geq DsoVoltageLevel$ &  DYNModelLoadRestorativeWithLimits \\
\bottomrule
\end{tabular}
\caption{Loads modelling}
\end{table}

\subsection{Generators}

The following parameters from the configuration file are used for generators modelling (see section \ref{Dynaflow_Launcher_Configuration_Configuration_File}):
\begin{itemize}
  \item The parameter \textit{InfiniteReactiveLimits} determines whether the generator models have infinite reactive power limits or PQ diagrams;
  \item The parameter \textit{ActivePowerCompensation} determines whether the generators participate in the active power balancing proportionally to PMax, P or PTarget.
\end{itemize}

The information that a generator partipates or not in the voltage regulation is given in the IIDM input file.

\par Concerning the voltage regulation:

\begin{table}[h!]
\center
\begin{tabular}{ c | c | c}
\toprule
\footnotesize{\textbf{{Voltage regulation}}} & \footnotesize{\textbf{{InfiniteReactiveLimits}}} & \footnotesize{\textbf{{Model}}}\\
\midrule
\rowcolor{white}
 \small{None}  & \small{-} & \footnotesize{fixed PQ generator} \\
\rowcolor{gray!10}
 \small{At the connection bus} & \small{false} & \footnotesize{GeneratorPVDiagramPQSignalN} \\
\rowcolor{white}
 \small{At the connection bus} & \small{true} & \footnotesize{GeneratorPVSignalN} \\
\rowcolor{gray!10}
 \small{At a distant bus} & \small{false} & \footnotesize{GeneratorPVRemoteDiagramPQSignalN} \\
\rowcolor{white}
 \small{At a distant bus} & \small{true} & \footnotesize{GeneratorPVRemoteSignalN} \\
\rowcolor{gray!10}
\small{Another generator}&  &  \\
\rowcolor{gray!10}
\small{regulates the same node} & \multirow{-2}{*}{\small{false}} & \multirow{-2}{*}{\footnotesize{GeneratorPQPropDiagramPQSignalN}} \\
\rowcolor{white}
\small{Another generator}&  & \footnotesize{GeneratorPQPropSignalN}  \\
\rowcolor{white}
\small{regulates the same node} & \multirow{-2}{*}{\small{true}} & \footnotesize{+ VRRemote model} \\
\bottomrule
\end{tabular}
\caption{Generators modelling}
\end{table}

\par Concerning the active power balancing:

\begin{itemize}
  \item A generator that does not participate in the voltage regulation cannot participate in the acting power balancing;
  \item A generator with an active power target equal to 0 does not participate in the active power balancing. It can however regulate the voltage;
  \item the active power reference of the others generator define if they participate in the active power balancing.
\end{itemize}

\subsection{HVDC}

Two types of HVDC links are modelled in DynaFlow : LCCs and VSCs.

The parameter \textit{InfiniteReactiveLimits} from the configuration file determines whether the HVDC models have infinite reactive power limits
or PQ diagrams (see section \ref{Dynaflow_Launcher_Configuration_Configuration_File}).

\subsubsection{VSCs}

The conditions and models used are given below.
The information that a HVDC has an AC regulation law is given in the IIDM input file.\\

\begin{table}[ht!]
\center
\begin{tabular}{ c | c | c}
\toprule
\small{\textbf{{Voltage regulation}}}& \small{\textbf{{InfiniteReactiveLimits}}} & \small{\textbf{{Model}}} \\
\midrule
\rowcolor{gray!10}
 \small{At the connection bus} & \small{false}& \small{HvdcPVDiagramPQ} \\
\rowcolor{white}
 \small{At the connection bus} & \small{true}& \small{HvdcPV} \\
\rowcolor{gray!10}
 \small{At a distant bus} & \small{false} & \small{HvdcPQPropDiagramPQ} \\
\rowcolor{white}
 \small{At a distant bus} & \small{true} & \small{HvdcPQProp} \\
\bottomrule
\end{tabular}
\caption{VSCs modelling (not dangling, no AC emulation law)}
\end{table}

\begin{table}[ht!]
\center
\begin{tabular}{ c | c | c}
\toprule
\footnotesize{\textbf{{Voltage regulation}}}& \footnotesize{\textbf{{InfiniteReactiveLimits}}} & \small{\textbf{{Model}}} \\
\midrule
\rowcolor{gray!10}
 \footnotesize{At the connection bus} & \small{false}& \footnotesize{HvdcPVDiagramPQEmulation} \\
\rowcolor{white}
 \footnotesize{At the connection bus} & \small{true}& \footnotesize{HvdcPVEmulation} \\
\rowcolor{gray!10}
 \footnotesize{At a distant bus} & \small{false} & \footnotesize{HvdcPQPropDiagramPQEmulation} \\
\rowcolor{white}
 \footnotesize{At a distant bus} & \small{true} & \footnotesize{HvdcPQPropEmulation} \\
\bottomrule
\end{tabular}
\caption{VSCs modelling (not dangling, AC emulation law)}
\end{table}

\begin{table}[ht!]
\center
\begin{tabular}{ c | c | c }
\toprule
\footnotesize{\textbf{{Voltage regulation}}}& \footnotesize{\textbf{{InfiniteReactiveLimits}}} & \small{\textbf{{Model}}} \\
\midrule
\rowcolor{gray!10}
 \footnotesize{At the connection bus} & \small{false}& \footnotesize{HvdcPVDanglingDiagramPQ} \\
\rowcolor{white}
 \footnotesize{At the connection bus} & \small{true}& \footnotesize{HvdcPVDangling} \\
\rowcolor{gray!10}
 \footnotesize{At a distant bus} & \small{false} & \footnotesize{HvdcPQPropDanglingDiagramPQ} \\
\rowcolor{white}
 \footnotesize{At a distant bus} & \small{true} & \footnotesize{HvdcPQPropDangling} \\
\bottomrule
\end{tabular}
\caption{VSCs modelling (dangling)}
\end{table}

\subsubsection{LCCs}

The conditions and models used are given below.

\begin{table}[ht!]
\center
\begin{tabular}{ c | c | c }
\toprule
\small{\textbf{{Dangling}}}& \small{\textbf{{InfiniteReactiveLimits}}} & \small{\textbf{{Model}}} \\
\midrule
\rowcolor{white}
 \small{false} & \small{false}  & \small{HvdcPTanPhi} \\
\rowcolor{gray!10}
 \small{true} & \small{false}& \small{HvdcPTanPhiDangling} \\
\rowcolor{white}
 \small{false} & \small{true}& \small{HvdcPTanPhiDiagramPQ} \\
\rowcolor{gray!10}
 \small{true} & \small{true} & \small{HvdcPTanPhiDanglingDiagramPQ} \\
\bottomrule
\end{tabular}
\caption{LCCs modelling}
\end{table}

\subsection{Static Var Compensators}

The conditions and models used are given below.

\begin{table}[h!]
\center
\begin{tabular}{ c | c | c | c}
\toprule
& & \scriptsize{\textbf{{Custom mode}}} &  \\
\scriptsize{\multirow{-2}{*}{\textbf{{Voltage regulation}}}}& \multirow{-2}{*}{\scriptsize{\textbf{{Control law}}}}& \scriptsize{\textbf{{control law}}} & \multirow{-2}{*}{\scriptsize{\textbf{{Model}}}} \\
\midrule
\rowcolor{white}
 \scriptsize{At the connection bus} & \scriptsize{None}& \scriptsize{false}& \scriptsize{StaticVarCompensatorPV} \\
\rowcolor{gray!10}
 \scriptsize{At the connection bus} & \scriptsize{None}& \scriptsize{true}& \scriptsize{StaticVarCompensatorPVModeHandling} \\
\rowcolor{white}
 \scriptsize{At the connection bus} & \scriptsize{U + Lambda*Q}& \scriptsize{false}& \scriptsize{StaticVarCompensatorPVProp} \\
\rowcolor{gray!10}
 \scriptsize{At the connection bus} & \scriptsize{U + Lambda*Q}& \scriptsize{true}& \scriptsize{StaticVarCompensatorPVPropModeHandling} \\
\rowcolor{white}
 \scriptsize{At a distant bus} & \scriptsize{None}& \scriptsize{false}& \scriptsize{StaticVarCompensatorPVRemote} \\
\rowcolor{gray!10}
 \scriptsize{At a distant bus} & \scriptsize{None}& \scriptsize{true}& \scriptsize{StaticVarCompensatorPVRemoteModeHandling} \\
\rowcolor{white}
 \scriptsize{At a distant bus} & \scriptsize{U + Lambda*Q}& \scriptsize{false}& \scriptsize{StaticVarCompensatorPVPropRemote} \\
\rowcolor{gray!10}
 \scriptsize{At a distant bus} & \scriptsize{U + Lambda*Q}& \scriptsize{true}& \scriptsize{StaticVarCompensatorPVPropRemoteModeHandling} \\
\bottomrule
\end{tabular}
\caption{Static Var Compensators modelling}
\end{table}


\section{Solver}

The solver used is the Simplified Solver (SolverSIM) from \Dynawo.
The Maximum time step value is by default 10s. This value can be modified in the configuration file (see section \ref{Dynaflow_Launcher_Configuration_Configuration_File}).
Notice that the maximum step value should be reduced only when the simulation is including
Special Protection Scheme automatons with a timescale smaller than 10 seconds: this would allow one to properly
take into account the SPS behaviour, avoiding any artificial synchronization effect. When modifying the
solver maximum time step, its value must be included in $ 1.0  < TimeStep \leq 10.0 $ seconds.

The full configuration (with default values) used is the following:

\lstinputlisting[language=XML,title=\Dynawo Simplified Solver configuration]{../resources/syntaxExample/solver.par}

\end{document}

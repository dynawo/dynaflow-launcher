%% Except where otherwise noted, content in this documentation is Copyright (c)
%% 2022, RTE (http://www.rte-france.com) and licensed under a
%% CC-BY-4.0 (https://creativecommons.org/licenses/by/4.0/)
%% license. All rights reserved.

\documentclass[a4paper, 12pt]{report}

% Latex setup
\input{../latex_setup.tex}

\begin{document}

\title{Dynaflow-launcher Installation Documentation}
\date\today

\maketitle
\tableofcontents

\chapter{Install procedure}

Dynaflow-launcher is available on \textbf{Linux}.
You can either build it from sources or use binaries distributed on Github.
This project uses the \href{https://github.com/dynawo/dynawo}{\Dynawo}  and the \href{https://github.com/dynawo/dynawo-algorithms}{\Dynawo-algorithms} projects.

\section{Dynaflow-launcher binaries distribution}

Dynaflow-launcher release is available on Github : \href{https://github.com/dynawo/dynaflow-launcher/releases/download/v1.4.1/DynaFlowLauncher_Linux_v1.4.1.zip}{Linux release}.

The packages required to use the distribution are the same as \Dynawo.

\subsection{Using a distribution}

You can use the following commands to download and test the latest distribution:
\begin{lstlisting}[language=bash, breaklines=true, breakatwhitespace=false, columns=fullflexible]
$> curl -L $(curl -s -L -X GET https://api.github.com/repos/dynawo/dynaflow-launcher/releases/latest | grep "DynaFlowLauncher_Linux" | grep url | cut -d '"' -f 4) -o DynaflowLauncher_Linux_latest.zip
$> unzip DynaflowLauncher_Linux_latest.zip
$> cd dynaflow-launcher
$> ./dynaflow-launcher.sh --network tests/main/res/TestIIDM_launch.iidm --config tests/main/res/config_launch.json
$> ./dynaflow-launcher.sh help
\end{lstlisting}

\section{Building requirements}

Dynaflow-launcher is tested on Linux platforms (Centos, Debian and Ubuntu based) and provided that you can install system packages there should be no problem on any other Linux distribution.

The requirements to build Dynaflow-launcher are the same as \Dynawo.

\section[Building Dynaflow-launcher]{Building Dynaflow-launcher}
\label{Dynaflow_launcher_Installation_Documentation_Building_Dynaflow_launcher}
The first step is to build \Dynawo in a separate folder. Please refer to the \Dynawo documentation to do so.
Then, the following command needs to be launched from the \Dynawo folder.

\begin{lstlisting}[language=bash, columns=fullflexible]
$> ./myEnvDynawo.sh deploy
\end{lstlisting}

This command creates a deploy folder. The path to dynawo deploy is then the path to the subdirectory dynawo in the deploy folder. It is similar to:

\begin{lstlisting}[language=bash, columns=fullflexible]
$> PATH_TO_DYNAWO_DEPLOY=<DYNAWO FOLDER>/deploy/<COMPILER><COMPILER VERSION>/shared/dynawo/
\end{lstlisting}

The second step is to do the same for \Dynawo-algorithms. Please refer to the \Dynawo-algorithms documentation to do so.
Then, the following command needs to be launched from the \Dynawo-algorithms folder.

\begin{lstlisting}[language=bash, columns=fullflexible]
$> ./myEnvDynawoAlgorithms.sh deploy
\end{lstlisting}

This command creates a deploy folder. The path to dynawo deploy is then the path to the subdirectory dynawo in the deploy folder. It is similar to:

\begin{lstlisting}[language=bash, columns=fullflexible]
$> PATH_TO_DYNAWO_ALGORITHMS_DEPLOY=<DYNAWO-ALGORITHMS FOLDER>/deploy/<COMPILER><COMPILER VERSION>/shared/dynawo/
\end{lstlisting}


To build Dynaflow-launcher you need to clone the github repository and launch the following commands in the source code directory:

\begin{lstlisting}[language=bash, columns=fullflexible]
$> git clone https://github.com/dynawo/dynaflow-launcher.git
$> cd dynaflow-launcher
$> echo '#!/bin/bash

# Required
export DYNAFLOW_LAUNCHER_HOME=$(cd "$(dirname "${BASH_SOURCE[0]}")" && pwd)
export DYNAWO_HOME=<PATH_TO_DYNAWO_DEPLOY>
export DYNAWO_ALGORITHMS_HOME=<PATH_TO_DYNAWO_ALGORITHMS_DEPLOY>
export DYNAFLOW_LAUNCHER_BUILD_TYPE=Release

# Optional
# export DYNAFLOW_LAUNCHER_LOCALE=en_GB
# export DYNAFLOW_LAUNCHER_CMAKE_GENERATOR=Ninja # default is Unix Makefiles
# export DYNAFLOW_LAUNCHER_PROCESSORS_USED=8 # default 1
# export DYNAFLOW_LAUNCHER_USE_DOXYGEN=OFF # default ON
# export DYNAFLOW_LAUNCHER_BUILD_TESTS=OFF # default ON
# export DYNAFLOW_LAUNCHER_LOG_LEVEL=INFO # default INFO: can be DEBUG, INFO, WARN, ERROR
# export DYNAFLOW_LAUNCHER_BROWSER=firefox # browser command used to visualize test coverage. default: firefox

# Optional external links : optional variable used at runtime to use additional iidm extension
# export DYNAWO_IIDM_EXTENSION=<PATH_TO_IIDM_EXTENSIONS_LIBRARY>

# Run
$DYNAFLOW_LAUNCHER_HOME/scripts/envDFL.sh $@' > myEnvDFL.sh
$> chmod +x myEnvDFL.sh
$> ./myEnvDFL.sh build-user
\end{lstlisting}

Below is a description of some environment variables that can be modified in the file \textit{myEnvDFL.sh}:

\begin{center}
\begin{tabular}{|l|l|}
  \hline
   \tiny{DYNAFLOW\_LAUNCHER\_BROWSER} & \small{Default browser command} \\
  \hline
   \tiny{DYNAFLOW\_LAUNCHER\_PROCESSORS\_USED} & \small{Maximum number of cores to use} \\
  \hline
   \tiny{DYNAFLOW\_LAUNCHER\_BUILD\_TYPE} & \small{Build type: Release or Debug} \\
  \hline
   \tiny{DYNAFLOW\_LAUNCHER\_LOG\_LEVEL} & \small{Level of Dynaflow-launcher logs} \\
  \hline
   \tiny{DYNAFLOW\_LAUNCHER\_BUILD\_TESTS} & \small{Enable tests build} \\
  \hline
   \tiny{DYNAWO\_IIDM\_EXTENSION} & \small{Path to iidm extension library} \\
  \hline
\end{tabular}
\end{center}

\textbf{Warning}: If you're working behind a proxy make sure you have exported the following proxy environment variables:

\begin{lstlisting}[language=bash, columns=fullflexible]
$> export http_proxy=
$> export https_proxy=
$> export no_proxy=localhost,127.0.0.0/8,::1
$> export HTTP_PROXY=$http_proxy;export HTTPS_PROXY=$https_proxy;export NO_PROXY=$no_proxy;
\end{lstlisting}

\section[Launching Dynaflow-launcher]{Launching Dynaflow-launcher}

Once you have installed and compiled Dynaflow-launcher as explained in section \ref{Dynaflow_launcher_Installation_Documentation_Building_Dynaflow_launcher},
you can launch a simulation by calling one example:

\begin{lstlisting}[language=bash, breaklines=true, breakatwhitespace=false, columns=fullflexible]
$> ./myEnvDFL.sh tests/main/res/TestIIDM_launch.iidm tests/main/res/config_launch.json
\end{lstlisting}

This command launches Dynaflow on a simple network and should succeed if your installation went well and your compilation finished successfully.

\section{Third parties}

To run a simulation on Linux, Dynaflow-launcher uses external libraries that are downloaded and compiled during the building process:
\begin{itemize}
\item \href{https://www.mpich.org/}{\underline{MPICH}}, an implementation of the Message Passing Interface (MPI) standard distributed under a BSD-like license.
\Dynawo-algorithms is currently using the version 3.4.2.
\end{itemize}

To run a simulation on Windows, \Dynawo-algorithms uses an external librarie that has to be installed before the building process:
\begin{itemize}
\item \href{https://learn.microsoft.com/en-us/message-passing-interface/microsoft-mpi?redirectedfrom=MSDN}{\underline{MSMPI}}, a Microsoft implementation of the Message Passing Interface standard distributed under a MIT license.
\Dynawo-algorithms is currently using the version 10.1.2.
\end{itemize}

Finally, Dynaflow-launcher also uses others libraries for the unit testing process or to build its source documentation.

\end{document}

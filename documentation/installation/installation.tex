%% Except where otherwise noted, content in this documentation is Copyright (c)
%% 2022, RTE (http://www.rte-france.com) and licensed under a
%% CC-BY-4.0 (https://creativecommons.org/licenses/by/4.0/)
%% license. All rights reserved.

\documentclass[a4paper, 12pt]{report}

% Latex setup
%% Except where otherwise noted, content in this documentation is Copyright (c)
%% 2022, RTE (http://www.rte-france.com) and licensed under a
%% CC-BY-4.0 (https://creativecommons.org/licenses/by/4.0/)
%% license. All rights reserved.

% Latin Modern fam­ily of fonts
\usepackage{lmodern}

\usepackage[english]{babel}

% specify encoding
\usepackage[utf8]{inputenc} % input
\usepackage[T1]{fontenc} % output
\usepackage[table]{xcolor}

% Document structure setup
\usepackage{titlesec} % To change chapter format
\setcounter{tocdepth}{3} % Add subsubsection in Content
\setcounter{secnumdepth}{3} % Add numbering for subsubsection
\setlength{\parindent}{0pt} % No paragraph indentation

% Change title format for chapter
\titleformat{\chapter}{\Huge\bf}{\thechapter}{20pt}{\Huge\bf}

% To add links on page number in Content and hide red rectangle on links
\usepackage[hidelinks, linktoc=all]{hyperref}
\usepackage[nottoc]{tocbibind}  % To add biblio in table of content
\usepackage{textcomp} % For single quote
\usepackage{url} % Allow linebreaks in \url command
\usepackage{listings} % To add code samples

% Default listings parameters
\lstset
{
  aboveskip={1\baselineskip}, % a bit of space above
  backgroundcolor=\color{shadecolor}, % choose the background color
  basicstyle={\ttfamily\footnotesize}, % use font and smaller size \small \footnotesize
  breakatwhitespace=true, % sets if automatic breaks should only happen at whitespace
  breaklines=true, % sets automatic line breaking
  columns=fixed, % nice spacing -> fixed / flexible
  mathescape=false, % escape to latex false
  numbers=left, % where to put the line-numbers
  numberstyle=\tiny\color{gray}, % the style that is used for the line-numbers
  showstringspaces=false, % do not emphasize spaces in strings
  tabsize=4, % number of spaces of a TAB
  texcl=false, % activates or deactivates LaTeX comment lines
  upquote=true % upright quotes
}

% Avoid numbering starting at each chapter for figures
\usepackage{chngcntr}
\counterwithout{figure}{chapter}

\usepackage{tikz} % macro pack­age for cre­at­ing graph­ics
\usepackage{pgfplots} % draws func­tion plots (based on pgf/tikz)
\pgfplotsset{enlarge x limits=false, xlabel={\begin{small}$time$ (s)\end{small}}, height=0.6\textwidth, width=1\textwidth,
yticklabel style={text width={width("$-0.6$")},align=right},
/pgf/number format/precision=4}
\pgfplotstableset{col sep=semicolon}

\usepackage{algorithm} % Add algorithms
\usepackage[noend]{algpseudocode} %  all end ... lines are omitted in algos

\usepackage{amsmath} % Add math­e­mat­i­cal fea­tures
\usepackage{schemabloc} % Add block diagram library (french one)

\usepackage{adjustbox} % Add box for flowchart

\usepackage{booktabs} % for toprule and midrule in tables

\usepackage{tabularx}
\usepackage{multirow}

\usepackage[nolist]{acronym} % don’t write the list of acronyms.
% Acronyms list
\begin{acronym}
\acro{BDF}{Backward Differentiation Formula}
\acro{BE}{Backward Euler}
\acro{DAE}{Differential Algebraic Equations}
\acro{IDA}{Implicit Differential-Algebraic solver}
\acro{LLNL}{Lawrence Livermore National Lab}
\acro{KINSOL}{Krylov Inexact Newton SOLver}
\acro{NR}{Newton-Raphson}
\acro{PLL}{Phase-Locked Loop}
\acro{SVC}{Static Var Compensator}
\acro{SUNDIALS}{SUite of Nonlinear and DIfferential/ALgebraic equation Solvers}
\end{acronym}

% Syntax highlight
%% Except where otherwise noted, content in this documentation is Copyright (c)
%% 2022, RTE (http://www.rte-france.com) and licensed under a
%% CC-BY-4.0 (https://creativecommons.org/licenses/by/4.0/)
%% license. All rights reserved.

\usepackage{color}

\definecolor{blue}{rgb}{0,0,1}
\definecolor{lightblue}{rgb}{.3,.5,1}
\definecolor{darkblue}{rgb}{0,0,.4}
\definecolor{red}{rgb}{1,0,0}
\definecolor{darkred}{rgb}{.56,0,0}
\definecolor{pink}{rgb}{.933,0,.933}
\definecolor{purple}{rgb}{0.58,0,0.82}
\definecolor{green}{rgb}{0.133,0.545,0.133}
\definecolor{darkgreen}{rgb}{0,.4,0}
\definecolor{gray}{rgb}{.3,.3,.3}
\definecolor{darkgray}{rgb}{.2,.2,.2}
\definecolor{shadecolor}{gray}{0.925}

% **********************************************************************************
% Syntax : Bash (bash)
% **********************************************************************************

\lstdefinelanguage{bash}
{
  keywordstyle=\color{blue},
  morekeywords={
    cd,
    export,
    source},
  numbers=none,
  deletekeywords={jobs}
}

% **********************************************************************************
% Syntax : XML
% **********************************************************************************

\lstdefinelanguage{XML}
{
  morestring=[s][\color{purple}]{"}{"},
  morecomment=[s][\color{green}]{<?}{?>},
  morecomment=[s][\color{green}]{<!--}{-->},
  stringstyle=\color{black},
  identifierstyle=\color{blue},
  keywordstyle=\color{red},
  morekeywords={
    xmlns,
    xsi,
    noNamespaceSchemaLocation,
    type,
    source,
    target,
    version,
    tool,
    transRef,
    roleRef,
    objective,
    eventually}
}


% **********************************************************************************
% Syntax : JSON
% **********************************************************************************

\colorlet{punct}{red!60!black}
\definecolor{background}{HTML}{EEEEEE}
\definecolor{delim}{RGB}{20,105,176}
\colorlet{numb}{magenta!60!black}
\lstdefinelanguage{json}{
    basicstyle=\normalfont\ttfamily,
    numbers=left,
    numberstyle=\scriptsize,
    stepnumber=1,
    numbersep=8pt,
    showstringspaces=false,
    breaklines=true,
    frame=lines,
    backgroundcolor=\color{background},
    literate=
     *{0}{{{\color{numb}0}}}{1}
      {1}{{{\color{numb}1}}}{1}
      {2}{{{\color{numb}2}}}{1}
      {3}{{{\color{numb}3}}}{1}
      {4}{{{\color{numb}4}}}{1}
      {5}{{{\color{numb}5}}}{1}
      {6}{{{\color{numb}6}}}{1}
      {7}{{{\color{numb}7}}}{1}
      {8}{{{\color{numb}8}}}{1}
      {9}{{{\color{numb}9}}}{1}
      {:}{{{\color{punct}{:}}}}{1}
      {,}{{{\color{punct}{,}}}}{1}
      {\{}{{{\color{delim}{\{}}}}{1}
      {\}}{{{\color{delim}{\}}}}}{1}
      {[}{{{\color{delim}{[}}}}{1}
      {]}{{{\color{delim}{]}}}}{1},
}

% **********************************************************************************
% Syntax : Modelica (modelica)
% **********************************************************************************
\lstdefinelanguage{Modelica}{
  alsoletter={...},
  morekeywords=[1]{ % types
      Boolean,
      Integer,
      Real},
  keywordstyle=[1]\color{red},
  morekeywords=[2]{ % keywords
    algorithm,
    and,
    annotation,
    assert,
    block,
    class,
    connector,
    constant,
    discrete,
    else,
    elseif,
    elsewhen,
    end,
    equation,
    exit,
    extends,
    external,
    false,
    final,
    flow,
    for,
    function,
    if,
    in,
    inner,
    input,
    import,
    loop,
    model,
    nondiscrete,
    not,
    or,
    outer,
    output,
    package,
    parameter,
    public,
    protected,
    record,
    redeclare,
    replaceable,
    return,
    size,
    terminate,
    then,
    true,
    type,
    when,
    while},
  keywordstyle=[2]\color{darkred},
  morekeywords=[3]{ % functions
    abs,
    acos,
    asin,
    atan,
    atan2,
    Complex,
    connect,
    conj,
    cos,
    cosh,
    cross,
    der,
    edge,
    exp,
    fromPolar,
    imag,
    noEvent,
    pre,
    sign,
    sin,
    sinh,
    sqrt,
    tan,
    tanh},
  keywordstyle=[3]\color{blue},
  morecomment=[l][\color{green}]{//}, % comments
  morecomment=[s][\color{green}]{/*}{*/}, % comments
  morestring=[b][\color{pink}]{'}, % strings
  morestring=[b][\color{pink}]{"}, % strings
}

\usepackage{tikz}
\definecolor{blue}{rgb}{.3,.5,1}
\definecolor{red}{rgb}{1,0,0}
\usetikzlibrary{shapes,arrows}
% Define block styles
\tikzstyle{decision} = [diamond, draw, fill=blue!20,
    text width=4.5em, text badly centered, node distance=3cm, inner sep=0pt]
\tikzstyle{block} = [rectangle, draw, fill=blue!20,
    text width=5em, text centered, rounded corners, minimum height=4em]
\tikzstyle{line} = [draw, -latex']
\tikzstyle{cloud} = [draw, ellipse,fill=red!20, node distance=3cm,
    minimum height=2em]
    \usetikzlibrary{calc}


\usepackage{xspace} % Define typography
\usepackage{dirtree}
\newcommand{\Dynawo}[0]{Dyna$\omega$o\xspace}


\begin{document}

\title{Dynaflow-launcher Installation Documentation}
\date\today

\maketitle
\tableofcontents

\chapter{Install procedure}

Dynaflow-launcher is available on \textbf{Linux}.
You can either build it from sources or use binaries distributed on Github.
This project uses the \href{https://github.com/dynawo/dynawo}{\Dynawo}  and the \href{https://github.com/dynawo/dynawo-algorithms}{\Dynawo-algorithms} projects.

\section{Dynaflow-launcher binaries distribution}

Dynaflow-launcher release is available on Github : \href{https://github.com/dynawo/dynaflow-launcher/releases/download/v1.3.0/Dynaflow_Launcher_Linux_v1.3.0.zip}{Linux release}.

The packages required to use the distribution are the same as \Dynawo.

\subsection{Using a distribution}

You can use the following commands to download and test the latest distribution:
\begin{lstlisting}[language=bash, breaklines=true, breakatwhitespace=false]
$> curl -L $(curl -s -L -X GET https://api.github.com/repos/dynawo/dynaflow-launcher/releases/latest | grep "Dynaflow_Launcher_Linux" | grep url | cut -d '"' -f 4) -o Dynaflow_Launcher_Linux_latest.zip
$> unzip Dynaflow_Launcher_Linux_latest.zip
$> cd dynaflow-launcher
$> ./dynaflow-launcher.sh --network tests/main/res/TestIIDM_launch.iidm --config tests/main/res/config_launch.json
$> ./dynaflow-launcher.sh help
\end{lstlisting}

\section{Building requirements}

Dynaflow-launcher is tested on Linux platforms (Centos, Debian and Ubuntu based) and provided that you can install system packages there should be no problem on any other Linux distribution.

The requirements to build Dynaflow-launcher are the same as \Dynawo.

\section[Building Dynaflow-launcher]{Building Dynaflow-launcher}
\label{Dynaflow_launcher_Installation_Documentation_Building_Dynaflow_launcher}
The first step is to build \Dynawo in a separate folder. Please refer to the \Dynawo documentation to do so.
Then, the following command needs to be launched from the \Dynawo folder.

\begin{lstlisting}[language=bash]
$> ./myEnvDynawo.sh deploy
\end{lstlisting}

This command creates a deploy folder. The path to dynawo deploy is then the path to the subdirectory dynawo in the deploy folder. It is similar to:

\begin{lstlisting}[language=bash]
$> PATH_TO_DYNAWO_DEPLOY=<DYNAWO FOLDER>/deploy/<COMPILER><COMPILER VERSION>/shared/dynawo/
\end{lstlisting}

The second step is to do the same for \Dynawo-algorithms. Please refer to the \Dynawo-algorithms documentation to do so.
Then, the following command needs to be launched from the \Dynawo-algorithms folder.

\begin{lstlisting}[language=bash]
$> ./myEnvDynawoAlgorithms.sh deploy
\end{lstlisting}

This command creates a deploy folder. The path to dynawo deploy is then the path to the subdirectory dynawo in the deploy folder. It is similar to:

\begin{lstlisting}[language=bash]
$> PATH_TO_DYNAWO_ALGORITHMS_DEPLOY=<DYNAWO-ALGORITHMS FOLDER>/deploy/<COMPILER><COMPILER VERSION>/shared/dynawo/
\end{lstlisting}


To build Dynaflow-launcher you need to clone the github repository and launch the following commands in the source code directory:

\begin{lstlisting}[language=bash]
$> git clone https://github.com/dynawo/dynaflow-launcher.git
$> cd dynaflow-launcher
$> echo '#!/bin/bash

# Required
export DYNAFLOW_LAUNCHER_HOME=$(cd "$(dirname "${BASH_SOURCE[0]}")" && pwd)
export DYNAWO_HOME=<PATH_TO_DYNAWO_DEPLOY>
export DYNAWO_ALGORITHMS_HOME=<PATH_TO_DYNAWO_ALGORITHMS_DEPLOY>
export DYNAFLOW_LAUNCHER_BUILD_TYPE=Release

# Optional
# export DYNAFLOW_LAUNCHER_LOCALE=en_GB
# export DYNAFLOW_LAUNCHER_CMAKE_GENERATOR=Ninja # default is Unix Makefiles
# export DYNAFLOW_LAUNCHER_PROCESSORS_USED=8 # default 1
# export DYNAFLOW_LAUNCHER_SHARED_LIB=ON # default OFF
# export DYNAFLOW_LAUNCHER_USE_DOXYGEN=OFF # default ON
# export DYNAFLOW_LAUNCHER_BUILD_TESTS=OFF # default ON
# export DYNAFLOW_LAUNCHER_LOG_LEVEL=INFO # default INFO: can be DEBUG, INFO, WARN, ERROR
# export DYNAFLOW_LAUNCHER_BROWSER=firefox # browser command used to visualize test coverage. default: firefox

# Optional external links : optional variable used at runtime to use additional iidm extension
# export DYNAWO_IIDM_EXTENSION=<PATH_TO_IIDM_EXTENSIONS_LIBRARY>

# Run
$DYNAFLOW_LAUNCHER_HOME/scripts/envDFL.sh $@' > myEnvDFL.sh
$> chmod +x myEnvDFL.sh
$> ./myEnvDFL.sh build-user
\end{lstlisting}

Below is a description of some environment variables that can be modified in the file \textit{myEnvDFL.sh}:

\begin{center}
\begin{tabular}{|l|l|}
  \hline
   \tiny{DYNAFLOW\_LAUNCHER\_BROWSER} & \small{Default browser command} \\
  \hline
   \tiny{DYNAFLOW\_LAUNCHER\_PROCESSORS\_USED} & \small{Maximum number of cores to use} \\
  \hline
   \tiny{DYNAFLOW\_LAUNCHER\_BUILD\_TYPE} & \small{Build type: Release or Debug} \\
  \hline
   \tiny{DYNAFLOW\_LAUNCHER\_LOG\_LEVEL} & \small{Level of Dynaflow-launcher logs} \\
  \hline
   \tiny{DYNAFLOW\_LAUNCHER\_BUILD\_TESTS} & \small{Enable tests build} \\
  \hline
   \tiny{DYNAWO\_IIDM\_EXTENSION} & \small{Path to iidm extension library} \\
  \hline
\end{tabular}
\end{center}

\textbf{Warning}: If you're working behind a proxy make sure you have exported the following proxy environment variables:

\begin{lstlisting}[language=bash]
$> export http_proxy=
$> export https_proxy=
$> export no_proxy=localhost,127.0.0.0/8,::1
$> export HTTP_PROXY=$http_proxy;export HTTPS_PROXY=$https_proxy;export NO_PROXY=$no_proxy;
\end{lstlisting}

\section[Launching Dynaflow-launcher]{Launching Dynaflow-launcher}

Once you have installed and compiled Dynaflow-launcher as explained in section \ref{Dynaflow_launcher_Installation_Documentation_Building_Dynaflow_launcher},
you can launch a simulation by calling one example:

\begin{lstlisting}[language=bash, breaklines=true, breakatwhitespace=false]
$> ./myEnvDFL.sh tests/main/res/TestIIDM_launch.iidm tests/main/res/config_launch.json
\end{lstlisting}

This command launches Dynaflow on a simple network and should succeed if your installation went well and your compilation finished successfully.

\end{document}

%% Except where otherwise noted, content in this documentation is Copyright (c)
%% 2022, RTE (http://www.rte-france.com) and licensed under a
%% CC-BY-4.0 (https://creativecommons.org/licenses/by/4.0/)
%% license. All rights reserved.

\documentclass[a4paper, 12pt]{report}

% Latex setup
\input{../latex_setup.tex}

\begin{document}

\title{Dynaflow-launcher Advanced Documentation}
\date\today

\maketitle
\tableofcontents

\chapter{Advanced documentation}

This chapter contains documentation on advanced features for readers that would like to have a deep look in Dynaflow-launcher.
It will explain:
\begin{itemize}
\item Dynaflow-launcher code organization (\ref{Dynaflow_Launcher_Advanced_Documentation_Code_Organization})
\end{itemize}

\section{Code organization}
\label{Dynaflow_Launcher_Advanced_Documentation_Code_Organization}

The Dynaflow-launcher source code \href{https://github.com/dynawo/dynaflow-launcher.git}
{\underline{repository}} is organized as follows:
\dirtree{%
.1 cmake.
.1 cpplint.
.1 documentation.
.1 etc.
.1 scripts.
.1 sources.
.2 Algo.
.2 Common.
.2 Inputs.
.2 Outputs.
.1 tests.
}
\begin{itemize}
\item the cmake folder contains all the files related to the general
configuration of the compilation process;
\item the cpplint directory contains the Python scripts needed to use cpplint;
\item the documentation directory contains the different latex documents used to
create this documentation;
\item the etc folder contains several constant files used during the compilation or at runtime;
\item the scripts folder contains some utilities script used during compilation or during tests;
\item the sources directory contains all the code related to Dynaflow-launcher itself. This is the most important directory;
\item the tests directory contains the testsuite used to check the correct
behaviour of the tool.
\end{itemize}

The source code is divided into different subdirectories corresponding to the different parts of the Dynaflow-launcher tool. They are:
\begin{itemize}
\item the Algo directory contains the code related to the analysis of the input network to properly generate the inputs for Dynaflow;
\item the Common directory contains all the code dealing with common feature and methods;
\item the Inputs directory contains the code related to the analysis of Dynaflow-launcher inputs;
\item the Outputs directory contains the code related to the generation of Dynaflow input files;
\end{itemize}

\end{document}

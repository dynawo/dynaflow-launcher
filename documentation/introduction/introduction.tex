%% Except where otherwise noted, content in this documentation is Copyright (c)
%% 2022, RTE (http://www.rte-france.com) and licensed under a
%% CC-BY-4.0 (https://creativecommons.org/licenses/by/4.0/)
%% license. All rights reserved.

\documentclass[a4paper, 12pt]{report}

% Latex setup
\input{../latex_setup.tex}

\begin{document}

\title{Dynaflow-launcher Introduction Documentation}
\date\today

\maketitle
\tableofcontents

\chapter{Introduction}

\section{What is Dynaflow-launcher?}

\textbf{Dynaflow-launcher is a utility tool used to easily run \href{https://dynawo.github.io/about/dynaflow}{\underline{Dynaflow}}
starting from a minimal set of inputs.} \\

It provides the following possibilities:
\begin{itemize}
  \item \textbf{Unitary simulations}: computation of the steady-state solution on a given network (IIDM format) with Dynaflow;
  \item \textbf{Systematic analysis}: assessment of the stability of a single base network subject to different events;
\end{itemize}

Dynaflow-launcher is licensed under the terms of the \href{http://mozilla.org/MPL/2.0}{\underline{Mozilla Public License, v2.0}}.
The source code is hosted into a \href{https://github.com/dynawo/dynaflow-launcher} {\underline{GitHub repository}}. \\

\section{Changes from previous versions}
\subsection{Changes from v1.6.0}

\begin{itemize}
\item Added the possibility to connect Hvdc lines to secondary voltage controls
\item Various fixes in selection of hvdc line models, VRRemote connexions and secondary voltage controls instanciation
\item Improved warning and errors
\item extended assembling and setting format with new type of nodes and new macros
\end{itemize}

\subsection{Changes from v1.5.0}

\underline{General:}

\begin{itemize}
\item Dynaflow-launcher is now available on windows
\item Miscellanous fixes in generators parameters
\end{itemize}

\underline{Platform integration:}
DynaFlow launcher integration into PowSybl was tested with the following versions:
\begin{itemize}
\item \href{https://github.com/powsybl/powsybl-core/releases/tag/v5.3.2}{powsybl-core v5.3.2}
\item \href{https://github.com/powsybl/powsybl-dynawo/releases/tag/v1.14.1}{powsybl-dynawo v1.14.1}
\end{itemize}

\subsection{Changes from v1.4.1}

\underline{New features:}

\begin{itemize}
\item SA: Add an option in the configuration file to start from an initial state binary file (dumpstate)
\item NSA: Allow having a different configuration in the same JSON file for N and SA
\item NSA: automatically read the N final state file in SA for initialization (instead of IIDM)
\end{itemize}

\underline{Outputs:}

\begin{itemize}
\item Properly initialize frozen0 parameter of VRemote and SVC
\item Use the same precision in dynawo and in the simplified solver to avoid unstabilities in the curves
\item Default stop time of N and SA in presence of a SVC is automatically increased to be sure the simulation is stabilized before the event
\item timeline is now exported in XML format (for coherency with others result files)
\end{itemize}

\underline{Bug fixes:}

\begin{itemize}
\item Fixed wrong MPI path used in distribution (which was causing crashes at runtime when MPI was installed on the system)
\item Fix crash when filtering automaton that needs to be removed as not fully functional with the current configuration
\item Make sure QNomAlt and SNon parameters are coherent in the SVC model and the generators in the SVC area
\item Dump an error if a simulation contains SVC areas and is using a flat start
\item Shunts initially disconnected in NODE\_BREAKER topology are taken into account in automaton connections (allowing the automatons to connect them during the simulation)
\end{itemize}

\underline{New features:}
\begin{itemize}
\item Improved modeling of Hvdcs
\item Added the possibility to model secondary voltage controls
\item Added the NSA flow that allows to launch first a N and then a systematic analysis based on the N outputs
\item Added an option to select between warm and flat initialization
\item Added an option to model transformers within generators models when voltage is above a threshold
\end{itemize}

\underline{Platform integration:}
DynaFlow launcher integration into PowSybl was tested with the following versions:
\begin{itemize}
\item \href{https://github.com/powsybl/powsybl-core/releases/tag/v5.2.0}{powsybl-core v5.2.0}
\item \href{https://github.com/powsybl/powsybl-dynawo/releases/tag/v1.13.0}{powsybl-dynawo v1.13.0}
\end{itemize}

\subsection{Changes from v1.4.0}

\underline{New features:}
\begin{itemize}
\item Improved modeling of Hvdcs
\item Added the possibility to model secondary voltage controls
\item Added the NSA flow that allows to launch first a N and then a systematic analysis based on the N outputs
\item Added an option to select between warm and flat initialization
\item Added an option to model transformers within generators models when voltage is above a threshold
\end{itemize}

\underline{Performance:}
\begin{itemize}
\item Improved solver default configuration
\end{itemize}

\subsection{Changes from v1.3.1}

\underline{Main changes:}
\begin{itemize}
\item Creation of a result file
\item Improvements of solver and jobs configuration for performance
\item default event time is now set at 10s
\end{itemize}


\subsection{Changes from v1.3.0}
First release.

\end{document}

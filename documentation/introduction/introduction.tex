%% Except where otherwise noted, content in this documentation is Copyright (c)
%% 2022, RTE (http://www.rte-france.com) and licensed under a
%% CC-BY-4.0 (https://creativecommons.org/licenses/by/4.0/)
%% license. All rights reserved.

\documentclass[a4paper, 12pt]{report}

% Latex setup
\input{../latex_setup.tex}

\begin{document}

\title{Dynaflow-launcher Introduction Documentation}
\date\today

\maketitle
\tableofcontents

\chapter{Introduction}

\section{What is Dynaflow-launcher?}

\textbf{Dynaflow-launcher is a utility tool used to easily run \href{https://dynawo.github.io/}{\underline{Dynaflow}}
starting from a minimal set of inputs.} \\

It provides the following possibilities:
\begin{itemize}
  \item \textbf{Unitary simulations}: computation of the steady-state solution on a given network (IIDM format) with Dynaflow;
  \item \textbf{Systematic analysis}: assessment of the stability of a single base network subject to different N-k events;
\end{itemize}

Dynaflow-launcher is licensed under the terms of the \href{http://mozilla.org/MPL/2.0}{\underline{Mozilla Public License, v2.0}}.
The source code is hosted into a \href{https://github.com/dynawo/dynaflow-launcher} {\underline{GitHub repository}}. \\

\section{Changes from previous versions}

\subsection{Changes from v1.3.0}
First release.

\end{document}
